\chapter{Derivation (HIGG5D2)}

Pre-selection is applied to both data and MC samples using the
derivation framework in order to reduce the xAOD sample size.
We use the HIGG5D2 derivation for our DxAOD sample production.
More information on the derivation framework can be found in the 
Higgs group's Twiki.\footnote{\url {https://twiki.cern.ch/twiki/bin/view/AtlasProtected/HSG2xAODMigration}} 

\iffalse

At this stage, the following event selection is applied:
\begin{itemize}
\item at least one lepton (electron or muon) with $p_{\text{T}}>6$
  GeV passing `Medium' identification criteria\footnote{The definition
    of `Medium' identification criteria is found in
    Section~\ref{sec:obj}.};
\item at least one large coned jet with distance parameter $R=1.0$
  (''large-R'' or ''fat'' jet) with $p_{\text{T}}>150$ GeV or one small-coned jet with distance
  parameter $R=0.4$ (''small-R'' jet) reconstructed by the anti-$k_t$
  algorithm with $p_{\text{T}}>100$ GeV or two small-coned jets with $p_{\text{T}}>25$ GeV
\end{itemize}
No trigger requirements are applied on the derivation level.

Moreover, objects which do not satisfy one of the following
requirements are removed from the event:
\begin{itemize}
\item track-particles associated with reconstructed electrons;
\item track-particles associated with reconstructed muons;
\item track-particles associated with $R=0.4$ jets with $p_{\text{T}}>15$ GeV and $|\eta|<2.8$;
\item track-particles associated with $R=1.0$ jets with $p_{\text{T}}>150$ GeV and $|\eta|<2.8$.
\end{itemize}

In addition, for MC samples within an event particles are kept in
accordance to one of two sets:
\begin{itemize}
\item hadrons, b-hadrons, leptons not from hadron decays;
\item BSM particles, bosons, BSM particle decay products, top partons
  and immediate decay products, leptons not from hadron decays.
\end{itemize}

The trimming of the large-R jets (see Section~\ref{sec:jet_def}) is
also performed during the derivation.
\fi
