\section{ $m_{hh}$ Selection as a Function of Signal Mass}
Table~\ref{tab:mhh_sig_cuts} shows the $m_{hh}$ cut as a function of signal mass. $m_{hh}$ cut is not applied for non-resonant signal. Figures~\ref{fig:300to1000} through ~\ref{fig:1100to3000} show the $m_{hh}$ distributions for all signal masses. 


\begin{figure}[!h]
\begin{center}
\includegraphics*[width=0.67\textwidth] {figures/mhh_cut/300_to_1000}
\caption[$m_{hh}$ distribution for mass between 300 and 1000.]{$m_{hh}$ distribution for mass between 300 and 1000.}
\label{fig:300to1000}
\end{center}
\end{figure}

\begin{figure}[!h]
\begin{center}
\includegraphics*[width=0.77\textwidth] {figures/mhh_cut/1100_to_3000}
\caption[$m_{hh}$ distribution for mass between 1100 and 3000.]{$m_{hh}$ distribution for mass between 1100 and 3000.}
\label{fig:1100to3000}
\end{center}
\end{figure}

%%%old text from conf note
\iffalse
At the time of unblinding request with 2015 data, we only had two selections defined for resonances - selection2000 for high mass and selection700 for low mass. For high mass selection and for masses $\geq 2000$~GeV, we had agreed to keep the $m_{hh} > 1800$ GeV requirement. For masses $\leq 2000$~GeV, we had agreed to keep the same signal efficiency as was obtained for m(700) with selection700. At that point, the efficiency was 60\%.
And we unblinded  selection700 with 2015 data. Hence, we are keeping the same $m_{hh}$ cut for m(700). However, the efficiency of the same cut has increased to approximately 70\%. It is technically very difficult to pin down the exact reason for this but here are the changes made since then. 
\begin{enumerate}
\item Analysis release has been changed from 20.1 to 20.7. 
\item CP recommendations have all been updated. 
\item $b$-tagging working point of 85\% now comes with higher purity and more light jet rejection. 
\item Additional trigger items in 2016 data. 
\item The new signal MC with full mass range are aMCatNLO +Herwig++ samples, which is different from what we earlier had - Madgraph+Herwig. Both use the same tunes. 
\end{enumerate} 

Given this, we have now computed approx. 75\% efficiency $m_{hh}$ cut for each new mass point. The detailed cut for each mass point appears in Table~\ref{fig:mhhtable}. Figures~\ref{fig:300to1000} through ~\ref{fig:1100to3000} show the $m_{hh}$ distributions for all signal masses. 

\begin{figure}[!h]
\begin{center}
\includegraphics*[width=0.67\textwidth] {figures/mhh_cut/mhh_cut_table}
\caption[$m_{hh}$ selection a function of mass.]{$m_{hh}$ selection a function of mass. NB: 300 and 400 need separate optimisation and in the interest of time, we are not including them in the final result.}
\label{fig:mhhtable}
\end{center}
\end{figure}
\fi
