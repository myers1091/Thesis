%%%%Use this as a filler to get the template working
%%Introduction
\chapter{Experimental Setup}
%ATLAS is a multipurpose detector positioned around one of the 4 primry interaction points of the Large Hadron Collider (LHC) in Geneva, Switzerland. The LHC collides protons at a center of mass energy of 13 TeV, for Run 2, at a rate of approximately 600MHz. 
\section{Hadronic Colliders}
Hadronic colliders use two beams of non-fundamental particles, typically proton-proton or proton-antiproton. These accelerators benefit from the larger mass of the particles when compared to lepton colliders. This results in smaller synchrotron radiation in circular accelerators, as the radiation is inversely proportional to mass. This allows hadronic colliders to have a much larger center of mass energy than leptonic colliders for the same size circular ring. \newline
\indent While hadronic colliders typically have larger collision energies, they also have significantly "messier" collisions. In leptonic colliders, the only final state particles come from the colliding particles. In hadronic colliders, not all of the constituents of the hadrons interact in the hard collision. This leads to many additional particles in the final state. Additionally, each subparticle only carries a portion of the hadron momentum, it is impossible to know the exact initial energy of the collision. 
\section{Overlapping Collisions}
\section{The Large Hadron Collider}
%Source
The Large Hadron Collider (LHC) is a 27 kilometer ring underneath the Franco-Swiss border. The LHC accelerates beams of protons (or ions) to a center of mass energy of up to 13 TeV(5 TeV) in two antiparallel beams around the ring. The particles are then collided at 4 primary interaction points each of which has a dedicated detector: ATLAS, CMS, ALICE, and LHCb.
\subsection{Detector Coordinates}
%Source
Within the ATLAS detector, the interaction point defined the origin of the coordinate system. The z-axis, the longitudinal axis, runs along the beam line, the positive x-axis points toward the center of the LHC ring, and the positive y-axis points toward the surface. The detector is also described in r, ${\eta}$, ${\phi}$ coordinates. With the transverse plane, the plane perpendicular to the beam line, being described by r and ${\phi}$. The radial coordinate, r, describes the distance from the beam line. The azimuthal angle, ${\phi}$, is the angle from the x-axis around the beam line. The final coordinate, ${\eta}$, is referred to pseudorapidity and is defined as ${\eta = -ln(tan(\frac{\theta}{2}))}$. With ${\theta}$ being the angle from the y-axis. The variable ${\Delta{R}=\sqrt{\eta^{2} + \phi^{2}}}$ is used to describe the distance between detector objects.
\section{Detector Overview}
%Source
\begin{figure}[h]
\begin{center}
\includegraphics*[width=0.60\textwidth] {figures/ATLAS_det}%Taken from http://iopscience.iop.org/article/10.1088/1748-0221/3/08/S08003/meta
\caption[The ATLAS detector]{The ATLAS detector}
\label{fig:ATLAS_det}
\end{center}
\end{figure}
The ATLAS detector, Figure~\ref{fig:ATLAS_det} is a general purpose detector and the largest on the LHC.  It is made up of concentric subsystems, each with a specialized task: the inner detector, which is responsible for measuring the charge and momentum of charged particles; the calorimeters, which are responsible for measuring the energy of different electromagnetic and hadronic particles; the muon spectrometer, which measures the momentum of minimum ionizing particles (MIP), like muons; and the magnet system, which is responsible for bending the charged particles in the detector, allowing their charge to be measured. The subdetectors feed into a vast Trigger and Data Acquisition (TDAQ) system that is responsible for selecting collision events with interesting characteristics.

\subsection{The Inner Detector}
%Source Inner detector TDR, I: https://cds.cern.ch/record/331063?ln=en
%Source Inner detector TDR, II: https://cds.cern.ch/record/331064?ln=en
%Source IBL TDR: https://cds.cern.ch/record/1291633?ln=en
\begin{figure}[h]
\begin{center}
\includegraphics*[width=0.60\textwidth] {figures/inner_3D}%Taken from ID TDR
\caption[Cross section of the Inner Detector.]{Cross section of the Inner Detector.}
\label{fig:ID_cs}
\end{center}
\end{figure}
%Tile: https://cds.cern.ch/record/2004868/files/ATL-TILECAL-PROC-2015-002.pdf
%Tile TDR: https://cds.cern.ch/record/331062?ln=en
%LAr TDR: https://cds.cern.ch/record/331061?ln=en
%LAr: https://www.physics.utoronto.ca/~krieger/procs/Krieger_NSS05_Proc.pdf
The inner detector (ID), Figure~\ref{fig:ID_cs}, is the closest system to the beam pipe. It contains 4 separate pieces. In order of distance from the beam pipe: The Insertable B-Layer (IBL), the Pixel Detectors, the Semiconductor Tracker (SCT), and the Transition Radiation Tracker (TRT). These subsystems work together to give charged particle tracking within the pseudorapidity range of ${|\eta| < 2.5}$. The inner detector is surrounded by a 2 T solenoid magnet, section ~\ref{ssec:mag}. The magnetic field causes charged particles to curve as they pass through the ID. The radius and direction of this curve give sign of the charge, positive or negative, along with a momentum measurement of the particle. The other task of the ID is vertexing, or determining if the transient particle came from the interaction point of a slightly displaced point. This is used to identify long lived particles, like bottom or charm quarks. This is discussed further in section ~\ref{ssec:btag}. \linebreak
\indent The IBL is the newest addition to the ID, being installed during the 2016 shutdown. It is place directly outside the beam pipe in order to maintain good vertexing and b tagging in increased pileup environments. In order to facilitate the insertion of the IBL, the beam pipe inner radius was decreased by 4 mm (from 29 mm to 25 mm). The IBL utilizes planar sensors, similar to the Pixel Detector, and 3D sensors, allowing electrons to interact to the bulk of the sensor as opposed to just the surface, and functions as a fourth pixel layer of the Pixel Detector.\linebreak
%Inner detector TDR Fig 3-1
\indent The Pixel Detectors are a network of high granularity, silicon pixels which measure the 2D position of passing charged particles. The silicon pixels are n-doped silicon wafers. A high voltage is applied to the wafer and when a charged particle passes through the silicon an electron hole pair is created. The electron drifts to the electrode and creates a signal that is read out by the electronics.  The Pixel detector barrel is divided into 3 cylindrical layers, the innermost layer is the B-layer, followed by Layer 1 and Layer 2. Each is covered in ${50\mu{m}}$ x ${300\mu{m}}$ silicon pixels. In order to ensure complete coverage, an end cap module is placed on each side of the barrel. The end-caps consist of 4 wheels, each with an inner and outer ring of trapezoid shaped silicon detectors. \linebreak
\indent The SCT is made of a barrel detector and two end-cap detectors. The barrel SCT has 4 cylindrical layers made up of pairs of 6.36 cm x 6.36 cm silicon crystals glued together along one side. The end-cap SCT tracker is made of rings of SCT modules with either silicon or galium arsenide. These rings are arranged into 9 wheels on each side of the barrel.\linebreak
\indent Outside of the silicon detectors lies the TRT. The TRT is a straw detector comprised of 50000, 4mm diameter straws in the barrel and 320000 radial straws in the end-caps. There are 420000 electronic channels, which give a spatial resolution of ${170\mu{m}}$ per straw. The straws are filled with various mixtures of xenon argon, carbon dioxide, tetrafluoromethane and nitrogen gas. When a charged particle passes through the TRT, they ionize the gas. The ionized gas is attracted to the oppositely charged straw and wire and produce a signal that is later amplified and read out. The xenon in the gas mixture allows for accurate particle identification from the transition radiation photon detection.  This gives a good discrimination between electrons and charged pions.

\subsection{Calorimeters}\label{ssec:calo}
%NOTE:::: Add general calo info
Outside of the solenoid magnet lies the calorimetry system. The calorimeters are responsible for measuring the energy of both charged and neutral particles, with the exception of MIPs and non-interacting particles such as neutrinos. The calorimeters can be broken into two distinct pieces, the liquid Argon calorimeter (LAr) and the tile calorimeter. \linebreak
\indent The Liquid Argon (LAr) calorimeter is a sampling calorimeter that is used for electromagnetic calorimetry for the entire range of acceptance (${|\eta{}|<4.8}$). It is also used for hadronic calorimetry for higher pseudorapidity ${1.4<|\eta{}|<4.8}$ In the central "barrel" of the calorimeter (${|\eta{}| < 1.4}$), is made up of 1024 lead-stainless-steel converters with copper-polyimde multilayer readout boards. The pates and readouts are arranged in an "accordion-shaped" geometry, Fig XXX. This allows for complete azimuthal coverage with no gaps, giving a constant electromagnetic energy resolution. In between the accordion layers, liquid argon is used as the active medium. The system is enclosed in a cryostat to maintain the temperature of the detector. The LAr barrel is divided radially into 4 sampling layers that are read out with a granularity of 0.1x0.1 ${\eta}$x${\phi}$ which are referred to as Trigger Towers. The granularity of the layers can be found in table XXXXX (LAR TDR table 1-2). The layer closest to the beamline is the Presampler. This layer sits inside of the cryostat and is responsible for  correcting for the energy loss in front of the calorimeter (the same is done in the endcap). Inside the cryostat, there are 3 additional layers, Fig XXXX. The thickness of the layers is often described in terms of radiation lengths ${\Xi_{0}}$. Where a radiation length is the distance a electron travels before it loses approximately 1/2 of it's energy to photon emission.  The front layer has a thickness of ${4.3\Xi_{0}}$, followed by the middle layer with a thickness of ${16\Xi_{0}}$ and the back layer of thickness ${2\Xi_{0}}$. Since the middle layer is the thickest, the bulk of the energy is absorbed in that layer. \linebreak 
\indent Forward from the barrel, there are two electromagnetic endcap (EMEC) wheels with a similar accordion structure to the barrel. One covering ${1.4 < |\eta{}| < 2.5}$ and one from ${2.5 < |\eta{}| < 3.2}$ Outside of the EMEC is the Hadronic endcap (HEC). This is also a copper-LAr sampling calorimeter. It has a simpler parallel plate design. Finishing out the LAr calorimeter is the Forward Calorimeter (FCal), which is contained in the endcap cryostat. This calorimeter is in the very forward region of the detector. In this region, the particle flux is very high, so a dense calorimeter is necessary to avoid energy leaking into other pieces of the detector. There are 3 layers in the FCAL, the first is made of copper and the other two are made of tungsten. They are matrices of metal with concentric tubes filled with Argon, see Fig XXXXX. TDR fig 1-9. \linebreak
\indent In the central region ${|\eta{}|<1.7}$, the tile calorimeter (TileCal) is responsible for the hadronic calorimetry. The TileCal is a sampling calorimeter with alternating iron plate absorbers and plastic scintillating tiles, the orientation can be seen in Fig XXXX (https://cds.cern.ch/record/2004868/files/ATL-TILECAL-PROC-2015-002.pdf, fig 2) . It has a fixed central barrel and 2 extended barrel sections that can be moved. The TileCal has a depth of ${7.4\lambda{}}$, where ${\lambda{}}$ is the nuclear interaction length, the mean distance a hadronic particle travels before it undergoes an inelastic interaction. The readout has the same granularity as the LAr trigger towers (0.1x0.1)
\subsection{Muon Detectors}
%Figure of muon cutaway: https://www.researchgate.net/figure/4-Cut-away-view-of-the-ATLAS-muon-system-from-Ref-3_fig10_254469099
%muon tdr: http://atlas.web.cern.ch/Atlas/GROUPS/MUON/TDR/pdf_final/mTDR.pdf
To detect muons, ATLAS uses four different technologies. For precision energy and position measurements, monitored drift tubes (MDT) and cathode strip chambers (CSC) are used. The CSCs are used in regions of high flux, where the MDTs are not suitable.  For the muon trigger system, a fast system is needed to keep up with the high collision rate of the LHC. In the central region, resistive plate chambers (RPC) are used, while in the forward region, where flux is higher, thin gap chambers (TGC) are used. The muon system, much like the ID utilize a magnetic field to determine the charge of passing particles. The magnet system is further discussed in Section~\ref{ssec:mag}.\linebreak
\indent The MDTs are made up of 6 parallel layers of cylindrical aluminum drift tubes with a tungsten-rhenium wires. The drift tubes are filled with a mixture of argon, nitrogen and methane. The tubes are assembled on a support of spacer and they are monitored for deformation by a built-in optical system, hence the \textbf{monitored} drift tubes. \linebreak
\indent While the MDTs are very good at precision measurements. However, they are not appropriate in areas whith the high rate counts (${> 200 Hz/cm^{2}}$) sue to their large diameter and high operating pressure. This is the case for the first layer of muon measurement with pseudorapidities of ${|\eta{}|>2.0}$. For this region, CSCs are the spectrometer of choice. CSCs are multiwire proportional chambers with a cathode strip readout. This gives good single and two track resolution in this high rate region.\linebreak
\indent In the barrel region, the muon trigger system employs RPCs, a low occupancy chamber with fast response. RPCs are gaseous parallel-plate detectors of Bakelite. The system can operate in two modes, avalanche and streamer.  In streamer mode, a large potential across the plates generates a discharge around the ionizing particle. For avalanche mode, a smaller potential difference and large signal amplification in the electornics allows for increased rate capability.\linebreak
%http://citeseerx.ist.psu.edu/viewdoc/download?doi=10.1.1.664.2817&rep=rep1&type=pdf
\indent Finally, in the end-cap of ATLAS, TGCs provide 2 important components. For the trigger system, TGCs have good timing resolution compared to the MDTs and can deal with a rate of up to 100 ${KHz/cm^{2}}$. For measurement, TGCs provide the azimuthal coordinate to compliment the bending coordinate from the MDTs. The TGCs are made up of anode wires and graphite cathodes in between layers of fiberglass laminate. 
\subsection{Magnet System}\label{ssec:mag}
%https://cds.cern.ch/record/338080?ln=en


\subsection{Trigger System}
\subsubsection{L1 Trigger}
\subsubsection{High Level Trigger}
\section{Simulation}




