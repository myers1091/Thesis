%%%%Use this as a filler to get the template working
%%Introduction
\chapter{Introduction}
The Standard Model (SM) is the culmination of more than a century of work. The first piece added to the puzzle was the electron, discovered in 1891. Since then, 24 other particles have been discovered, with the final piece, the Higgs Boson, being added in 2012. Since it was theorized, the SM has held up to rigorous experimentation and remains an unbeaten theory of fundamental matter and forces. Even though the SM is widely successful, it fails to explain all observed phenomena. Gravity, neutrino masses, dark matter, along with other observations, all lacking explanation within the SM. The remaining task is to probe the extremes of the SM to either more precisely measure the parameters or to find its limit.
%Include this is a search for BSM production
\section{The Standard Model}
The Standard Model defines the basic building blocks of matter and force and the interactions between them. Normal matter that we interact with on a daily basis is made of protons, neutrons, and electrons. Electrons are a fundamental particle, called a lepton, meaning they are not made of smaller constituents. However, protons and neutrons are not fundamental particles. They are a composite of up and down quarks, two more fundamental particles. The protons are made of 2 ups and a down and the neutrons are made of two downs and an up. Leptons and quarks are both different types of fermions. \linebreak
\indent Fermions are spin-${\frac{1}{2}}$ particles that make up all matter in the SM. The fermions can be broken down into 3 "generations". Where a generation contains two quarks, one with electric charge ${+\frac{2}{3}}$ and one with electric charge ${-\frac{1}{3}}$, one electrically charged lepton, charge -1, and one electrically neutral lepton. The quarks have an additional color charge, of which there are 3 charges. This is additional quantum number associated with the strong force. In all, this gives 12 fermions. \linebreak
%bosons
\indent Gauge bosons are spin-${1}$ particles responsible for carrying the fundamental forces in the standard model. There are 12 physical gauge boson. The photon ${\gamma}$ is a massless, charge neutral force carrier for the electromagnetic force. The nuclear forces are carried by 3 massive gauge bosons. A chargeless Z boson and two charged W bosons, ${Q = \pm 1}$. Together, these 4 bosons control the electroweak interactions in the standard model. The remaining 8 bosons are the gluons, the force carriers for the strong nuclear interaction. Gluons are massless, electrically neutral particles that have two color charges. There is a gluon for each combination of the three color charges, giving the 8 total gluons. \linebreak
\indent The remaining piece of the standard is the Higgs Boson. The Higgs boson is a massive Scalar, spin-${0}$, chargeless boson. The Higgs boson is responsible for giving mass to the massive electoweak bosons through electroweak symmetry breaking.
\subsection{The Higgs Boson}
%Start with need massless gauge bosons to fufill local gauge invariance. So we have W 1,2,3, and b. explain the mixing and the  break the symmetry to give the W+- Z and photon. Give math to explain this?? Show how this gives rise to a new boson, the higgs. 
Electroweak field theory is a gauge invariant theory. This means the Lagrangian that describes the system is invariant under local gauge transformations. For the electroweak theory, this is the electroweak symmetry. To satisfy this symmetry, the bosons must be massless. However, the electroweak bosons in the standard model, the ${W^{\pm}}$ the ${Z}$ and the ${\gamma}$ are not all massless. This means that the electroweak symmetry must be broken by something.\linebreak
\indent In Electroweak theory, the five gauge bosons are ${W^{i}_{\mu}, i = 1,2,3}$ and ${B_{\mu}}$. These bosons couple to a complex scalar Higgs doublet, ${\Phi \equiv \binom{\phi^{+}}{\phi^{0}}}$. This doublet has a scalar potential.
\begin{equation}
V(\Phi) = \mu^{2}|\Phi^{\dagger}\Phi| + \lambda(|\Phi^{\dagger}\Phi|)^{2}
\end{equation}
Where ${\mu^{2} < 0}$. This gives the Mexican hat shaped potential seen in Fig(higgs potential plot, with a minimum energy at 
\begin{equation}
\langle \phi \rangle = \sqrt{-\frac{\mu^{2}}{2\lambda}}\equiv \frac{\nu}{\sqrt{2}}
\end{equation}
called the vacuum expectation value (VEV) of ${\phi}$. The choice of the direction of fluctuation is arbitrary but can be chosen such that 
\begin{equation}
\phi_{0} = \frac{1}{\sqrt{2}} \binom{0}{\nu}
\end{equation}
After the direction is chosen and the only remaining piece is the scalar field h(x), giving 
\begin{equation}
\phi(x) = \phi_{0} + h(x)
\end{equation}
The doublet can now be described by 
\begin{equation}
\Phi = \frac{1}{sqrt{2}} \binom{0}{v+h(x)}
\end{equation}
The Higgs field couples to the gauge bosons as 
\begin{equation}
(\frac{g}{2}\overrightarrow{\tau}\cdot \overrightarrow{W} + \frac{g'}{2}B)\phi_{0}
\end{equation}
Where ${\overrightarrow{\tau}}$ are the Pauli matrices, ${\overrightarrow{W}}$ are ${W_{1,2,3}}$ and g, g' are the coupling constants. The result of the coupling is the acquisition of mass by three eigenstates of the bosons, the ${W^{\pm}_{mu}}$ and the ${Z^{\mu}}$ and one massless eigenstate ${A^{\mu}}$, the photon. These four eigenstates are the bosons we observe in the standard model. 