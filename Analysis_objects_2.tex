
The final state of this analysis includes electrons, muons, neutrinos and jets, including $b$-jets. 
The identification criteria and the selection applied to the reconstructed objects are defined in the
present section.

\subsection{Electrons}
\label{sec:el_def}

\subsubsection{Electron reconstruction}
\label{sec:el_reco}
%Electromagnetic (EM) clusters are reconstructed with a sliding window algorithm. EM clusters are associated with 
%a track refitted with GSF (Gaussian Sum Filter model)~\cite{ATLAS-CONF-2012-047} to account for bremsstrahlung energy losses. 
%%No vertex from conversion is required for the EM cluster of the electron candidate.
% 
%Electron identification is performed using a likelihood-based method. Variables used by the likelihood identification algorithm are the longitudinal 
%and transverse shower profiles, the track quality, the track and cluster positions to match in $\eta$ and $\phi$ and the presence 
%of high-threshold TRT hits. 
%The likelihood-based method includes most of the discriminating variables 
%that are difficult to use with explicit requirement without incurring significant efficiency loss. 



For this analysis, two set of electron selections are defined. They are denoted as VHLooseElectron and SignalElectron.
The selections are defined as the following

\textbf{VHLooseElectron}: The electron \pt~is required to be greater than 7 GeV. 
The electron cluster should be in the range of $|\eta|< 2.47$. 
Loose likelihood identification is applied in this criteria. 
Impact parameter significance ($|d_{0}^{\mathrm{sig}}| = d_{0}/\sigma{_{d_{0}}}$) less than 10 standard deviations. 
and $|\Delta{z_{0}^{\mathrm{IBL}}}\sin\theta| < 0.5$ mm are also required, where IBL refers to the ATLAS Insertable $B$-Layer. 

\textbf{SignalElectron}: The electron is required to pass the VHLooseElectron selection with its \pt~required to be greater than 27 GeV. 
The electron cluster should be in the range of $|\eta|< 2.47$ but excluded from the crack region ($1.37 < |\eta| < 1.52$).
Tight likelihood identification is applied in SignalElectron criteria with the impact parameter significance required to be 
less than 2. In addition, the electron is required to be isolated by passing the \texttt{FixedCutTightTrackOnly} 
isolation working point which corresponds to a cut on the ratio of ${p_{T}^{\mathrm{varcone0.2}}}$ to electron \pt of 0.06 (i.e ${p_{T}^{\mathrm{varcone0.2}}/ \pt < 0.06}$).

A summary of the electron selections is shown in Table~\ref{tab:electronsel}.

\begin{table}[htbp!]
\begin{adjustbox}{width=1\textwidth}
\centering
\begin{tabular}{ccccccc} \hline \hline
Electron Selection & \pt & $|\eta|$ & ID & $|d_{0}^{\mathrm{sig}}|$ &  $|\Delta{z_{0}^{\mathrm{IBL}}}\sin\theta|$ & Isolation \\ \hline
VHLoooseElectron   & $>$7~GeV  & $< 2.47$ & LH Loose & $ <10$ & $<0.5$ mm & - \\
SignalElectron     & $>$27~GeV & $< 2.47$ and $\notin [1.37, 1.52]$ & LH Tight & $  <2$ & $<0.5$ mm & \texttt{FixedCutTightTrackOnly} \\
\hline\hline
\end{tabular}
\end{adjustbox}
\caption{Electron selection requirements.}\label{tab:electronsel}
\end{table}

\subsection{Muons}
\label{sec:mu_def}
 
\subsubsection{Muon reconstruction}
\label{sec:mu_reco}
%Muon candidates are identified by using the algorithm described in
%Ref.~\cite{Muons2015}. Muons are selected within $|\eta| < 2.5$
%using track quality criteria based on the number of hits in the
%inner detector and in the muon spectrometer. Medium quality criteria 
%are used for muon identification. The criteria include muons
%identified in both the inner detector and muon spectrometer with
%good matching of the two tracks for $|\eta| < 2.5$.


For this analysis, two sets of muon selections are defined. They are denoted as VHLooseMuon and SignalMuon.
The selections are defined as the following:

\textbf{VHLooseMuon}: The muon \pt~is required to be greater than 7 GeV. 
The muon cluster should be in the range of $|\eta|< 2.7$. 
Loose identification is applied in this criteria. 
Impact parameter significance ($|d_{0}^{sig}|$) less than 6 standard deviations. 
and $|\Delta{z_{0}^{\mathrm{IBL}}}\sin\theta| < 0.5$ mm are also required. 

\textbf{SignalMuon}: The muon is required to pass the VHLooseMuon selection with its \pt~required to be greater than 27 GeV 
and should be in the range of $|\eta|< 2.4$. Medium identification is applied in SignalMuon criteria with the 
impact parameter significance required to be less than 2. In addition, the muon is required to be isolated by passing 
the \texttt{FixedCutTightTrackOnly} isolation working point which corresponds to a cut on the ratio of ${p_{T}^{\mathrm{varcone0.3}}}$ to 
muon \pt of 0.06 (i.e ${p_{T}^{\mathrm{varcone0.3}}/ \pt < 0.06}$).

A summary of the muon selections is shown in Table~\ref{tab:muonsel}.

\begin{table}[htbp!]
\begin{adjustbox}{width=1\textwidth}
\centering
\begin{tabular}{ccccccc} \hline \hline
Muon Selection & \pt & $|\eta|$ & ID & $|d_{0}^{\mathrm{sig}}|$ & $|\Delta{z_{0}^{\mathrm{IBL}}}\sin\theta|$ & Isolation \\ \hline
VHLoooseMuon   & $>$7 GeV  & $ < 2.7$ & Loose quality  & $ <6$ & $<0.5$ mm & - \\
SignalMuon     & $>$27 GeV & $ < 2.4$ & Medium quality & $ <2$ & $<0.5$ mm & \texttt{FixedCutTightTrackOnly} \\
\hline\hline
\end{tabular}
\end{adjustbox}
\caption{Muon selection requirements.}
\label{tab:muonsel}
\end{table}

\subsection{Jets}
\label{sec:jet_def}
\subsubsection{Large-R jets}
For signal processes with a large resonant mass the b-jets produced by the Higgs may be too close together
to be resolved by the R=0.4 calorimeter-based jets (calo-jets). This effect is expected to be noticeable when ${p_{T}^{H} > 500\mathrm{GeV}}$\footnote{Using the rule of thumb ${}\Delta R = 2m/\pt$, where ${m = m_{H}}$ and ${\Delta R = 0.4}$}.
Our approach to reconstructing the ${H\rightarrow b\overline{b}}$ system in this ``boosted" regime is to use a large radius (large-R) jet with radius parameter R =1.0. The large-R jets are required to have ${\pt > 250 \mathrm{GeV}}$ and ${|\eta| < 2.0}$.
\subsubsection{Track jets}
To identify a large-R jet that is consistent with decay of ${H\rightarrow b\overline{b}}$, a method developed by ATLAS is to reconstruct subjets within the large-R jet and identify the subjets whether it is a b-jet or not by using a b-tagging algorithm. The baseline method is to use subjets built from tracks (track jets).
For the boosted analysis, track jets are required to have ${\pt > 10 \mathrm{GeV}}$ and ${|\eta| < 2.5}$ for them to be within the inner detector acceptance. They are also required to have at least 2 track constituents. The MV2c10 working point for track jets is the 77\% Fixed Cut efficiency.

\subsubsection{Small-R jets}


\textbf{Signal jets} are defined as jets which passes the jet cleaning and JVT criteria, described in the previous section.
They are further required to have $\pt > 20$ GeV and $|\eta| < 2.5$.

%\newcommand{\BTagWPFootNote}{The charm quark component is suppressed by a factor 3.10 while the 
%light quark component is suppressed by a factor 33.5. The expected performance is documented 
%on the \href{https://twiki.cern.ch/twiki/bin/view/AtlasProtected/BTaggingBenchmarks\#MV2c10_tagger_added_on_11th_May}{BTaggingBenchmarks twiki}.}

%The ATLAS jet flavor tagging algorithm, here the \texttt{MV2c10} algorithm~\cite{ATL-PHYS-PUB-2016-012}, is used to select signal jets and suppress multi-jet, $W$+jets, $Z$+jets and di-boson background. 
%Out of the possible working points corresponding to different $b$-tagging efficiencies, the 85\% Fixed-Cut\footnote{\BTagWPFootNote} working point (WP) is selected as to keep the signal efficiency high. 
Signal jets are labeled \textbf{b-jets} if they pass the \texttt{MV2c10} 85\% WP cut and labeled as \textbf{light-jets} if they fail the cut.



Table~\ref{tab:sjdefinit} summarizes the jets selection. 

\begin{table}[htbp!]
\centering 
\small
\begin{tabular}{|c||c|}        
 \hline
 & Signal Jets\\
 \hline
 Algorithm            & anti$-k_t$\\
 $p_T$                & 20~GeV\\
 $|\eta|$             & $< 2.5 $\\
 Quality              & not ``bad'' jet\\
 Pile-up jet removal & JVT $> 0.59$ when $|\eta| < 2.5 ~ and ~p_T < 60 $ GeV\\    
 $b$-tagging          &  \texttt{MV2c10}, 85\% fixed-cut WP, labelled as b-jets pass cut, light-jets if fail cut\\ 
\hline                          
\end{tabular}
\caption{Selection for jets with distance parameter $R = 0.4$.}
\label{tab:sjdefinit}
\end{table}

\subsection{Missing transverse momentum ($\met$)}
\label{sec:met_def}

%The neutrino is not directly detectable and, thus, appears only as an imbalance in
%transverse momentum.
%\footnote{Transverse momentum is defined as the components of
%momentum in the plane perpendicular to the beam axis.}.
	
The missing transverse momentum (MET, or \met)~\cite{ATL-PHYS-PUB-2015-027} used in this analysis is computed by using electrons that pass the VHLooseElectron selection, muons passing the VHLooseMuon selection and jets of the analysis.\footnote{From MET\_Core\_AntiKt4EMTopo with the MissingETAssociationMap using the METMaker tool. All calibrated jets are passed to the METMaker tool as prescribed on the EtMiss subgroup \apkg{https://twiki.cern.ch/twiki/bin/view/AtlasProtected/EtmissSubgroup}{twiki}} The track-based soft term\footnote{Defined on this \href{https://twiki.cern.ch/twiki/bin/view/AtlasProtected/EtmissSubgroupTrackSoftTermDescription}{twiki}.} (TST) is the recommended soft term component for the MET calculation. Photons and hadronically decaying taus are included in the \met{} calculation as jets since they are not used explicitly in the event reconstruction.

\subsection{Overlap removal}
\label{sec:overlapremoval}
Each object identification algorithm in ATLAS runs independently. This means the same physical object can be identified as multiple things. One example is electrons and jets, both of which can have tracks and EM calorimeter signatures. In order to uniquely identify objects, overlapping objects are removed according to the overlap removal procedure defined in this section. 
Electrons and muons that pass the VHLooseElectron and VHLooseMuon selections (as defined in Sec.~\ref{sec:el_reco} and ~\ref{sec:mu_reco}) are considered for overlap removal. 
Calorimeter jets which pass the JVT requirement are also considered for overlap removal. The procedure is defined as follows.

If an electron and a muon shares a track, the muon is removed if it is \textit{calo-tagged}. Otherwise, the electron is removed.
Calorimeter jets are then removed if they are within $\Delta R(\text{calo-jet}, \text{electron})$ < 0.2 of surviving electrons. 
Electrons that satisfy $\Delta R(\text{electron},\text{calo-jet})$ < min(0.4, 0.04 + 10 \GeV /$E^\text{electron}_\text{T}$) are removed. 
The surviving calorimeter jets are removed if they are within $\Delta R(\text{calo-jet}, \text{muon}) < 0.2$ and 
do \textbf{not} pass any of the following criteria:

\begin{itemize}
\item The number of tracks in the jet are more than 2.
\item $\pt^{\text{muon}} / \pt^{\text{calo-jet}} < 0.5$  AND  $\pt^{\text{muon}} / \pt^{\text{tracks in calo-jet}} < 0.7$.
\end{itemize}

Muons that satisfy $\Delta R(\text{muon},\text{calo-jet})$ < min(0.4, 0.04 + 10 \GeV /$\pt^\text{muon}$) are removed. 
The overlap removal procedure is implemented using ASG's 
\apkg{https://svnweb.cern.ch/trac/atlasoff/browser/PhysicsAnalysis/AnalysisCommon/AssociationUtils/trunk/doc/README.rst}{AssociationUtils} 
package and summarized in Table~\ref{tab:overlapremoval}.

\begin{table}[htbp!]
\centering
\tiny
\begin{tabular}{l | l}
\toprule
Overlapping Objects & Removal Procedure \\
\midrule
Electron - Muon   & If share track, remove muon if calo-tagged. Otherwise remove electron.\\ 
\hline
\multirow{2}{*}{Electron - Calo-jet} & If $\Delta R(\text{calo-jet}, \text{electron})$ < 0.2, remove calo-jet.\\
& If $\Delta R(\text{electron},\text{calo-jet})$ < min(0.4, 0.04 + 10 \GeV /$E^\text{electron}_\text{T}$), remove electron.\\ 
\hline
\multirow{4}{*}{Muon - Calo-jet} & If $\Delta R(\text{calo-jet}, \text{muon})$ < 0.2, remove calo-jet if: \\
& a) Number of tracks in calo-jet $\leq$ 2, OR \\  
& b) $\pt^{\text{muon}} / \pt^{\text{calo-jet}} > 0.5$  AND  $\pt^{\text{muon}} / \pt^{\text{tracks in calo-jet}} > 0.7$.\\
& If $\Delta R(\text{muon},\text{calo-jet})$ < min(0.4, 0.04 + 10 \GeV /$\pt^\text{muon}$), remove muon.\\ 
\bottomrule
\end{tabular}
\caption{A summary of the overlap removal procedure.} 
\label{tab:overlapremoval}
\end{table}
