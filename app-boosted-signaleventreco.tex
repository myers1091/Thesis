In this appendix, we present studies on the event reconstruction of the signal topology.

\subsection{Signal efficiency} 

Figure~\ref{fig:boosted_signaleff} shows the acceptance $\times$ efficiency as a function of signal mass points. The range of 
acceptance $\times$ efficiency varies for each signal mass point with roughly 2.5\% to be the highest around 1500 \GeV. The maximum 
acceptance $\times$ efficiency for each of the lepton channel is 15\%.


\begin{figure}[!htbp]
\begin{center}
\includegraphics*[width=0.45\textwidth]{./figures/boosted/signaleff_muon}
\includegraphics*[width=0.45\textwidth]{./figures/boosted/signaleff_elec}\\
\includegraphics*[width=0.45\textwidth]{./figures/boosted/signaleff_lepton}
\caption{Acceptance $\times$ efficiency of $H\rightarrow hh \rightarrow bbWW$ events at different 
selection levels of the analysis as a function of resonant signal mass point in the muon channel (top left), 
electron channel (top right) and combined lepton channel (bottom).}
\label{fig:boosted_signaleff}
\end{center}
\end{figure}


Figure~\ref{fig:boosted_signalreleff} shows the relative efficiency of each selection with respect to the preceding selection. The figure
shows that the b-tagging requirement on the two track-jets (purple line, star marker) has the lowest efficiency compare to other selections.

\begin{figure}[!htbp]
\begin{center}
\includegraphics*[width=0.45\textwidth]{./figures/boosted/signaleff_muon_releff}
\includegraphics*[width=0.45\textwidth]{./figures/boosted/signaleff_elec_releff}\\
\includegraphics*[width=0.45\textwidth]{./figures/boosted/signaleff_lepton_releff}
\caption{Relative efficiency of each selection as a function of resonant signal mass point
in the muon channel (top left), electron channel (top right) and combined 
lepton channel (bottom). Each selection efficiency is relative to the preceding selection.
The " >= 1 Large-$R$ " jet selection efficiency is relative to the " >= 1 Lepton " selection.}
\label{fig:boosted_signalreleff}
\end{center}
\end{figure}

\FloatBarrier

\subsection{$h \rightarrow b\bar{b}$ reconstruction} 

As the mass of the resonant signal particle increases, the daughter Higgs bosons from the decay of the resonant are produced with increasing
transverse momentum. We investigated the efficiency of jet reconstruction to reconstruct the $h \rightarrow b\bar{b}$ decay with detector-level studies 
guided by truth-level information. In this study, we looked at two different jet reconstruction strategy to reconstruct the $h \rightarrow b\bar{b}$. 
\begin{itemize}
 \item Resolved: Two $R=0.4$ calorimeter anti$-k_t$jets and each one of them contains exactly 1 $b$-hadron. This strategy aims to reconstruct $h \rightarrow b\bar{b}$. 
 system by identifying the tw
 o $b$-jets and reconstruct the Higgs boson kinematics from the two $b$-jets.
 \item Boosted:  One $R=1.0$ calorimeter anti$-k_t$ jet which contains a Higgs-boson and 2 $b$-hadrons and has at least two $R=0.2$ anti$-k_t$  track-jets 
 with the two highest \pt track-jets contains exactly 1 $b$-hadron. This strategy aims to reconstruct the Higgs boson kinematics by capturing all the decay products
 with one large-$R$ jet and use subjets to identify whether the large-$R$ jet is consistent with a $b\bar{b}$ decay.
\end{itemize}

Figure~\ref{fig:boosted_boostedvsresolved_hbb} shows the selection efficiency as a function of resonant signal mass 
with respect to " >= 1 Lepton " requirement,  which is after a reconstructed muon or electron is identified in the event. 
In the resolved channel (cyan line), the efficiency decreases as the resonant signal mass increases as the jet algorithm is less likely able to resolve
the two $b$-jets as they tend to be more collimated due to the increasing momentum of the parent Higgs boson.
As for the boosted channel, the efficiency to capture the Higgs boson decay products (i.e the two $b$-jets of the $b$-quarks) increases (red line) as 
the radius parameter of the $R=1.0$ is large enough. Requiring at least two track-jets (blue line) and the two highest \pt track-jets to have exactly 1 $b$-hadron 
each (green lines) decreases the efficiency as there will always be some proportion of track-jets that do not pass the minimum \pt cut and the track-jet reconstruction
are not able to resolve the two $b$-jets. The latter case is more prominent for high mass signals ($>2000~\GeV$) as it can be seen that the efficiency decreases as a function of resonant signal mass.

 
\begin{figure}[!htbp]
\begin{center}
\includegraphics*[width=0.48\textwidth]{./figures/boosted/BoostedVsResolved_truthmuon_recomuon_releff}
\includegraphics*[width=0.48\textwidth]{./figures/boosted/BoostedVsResolved_truthelec_recoelec_releff}
\caption{Relative efficiency of each selection with respect to " >= 1 Lepton " selection as a function of signal 
mass point in the muon channel (left) and the electron channel (right).
}
\label{fig:boosted_boostedvsresolved_hbb}
\end{center}
\end{figure}

\FloatBarrier