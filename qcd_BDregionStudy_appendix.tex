\section{Freezing B and D Regions in QCD Estimate}
\label{app:qcd_BDregionStudy_appendix}

This appendix summarizes the study undertaken to select regions to 'freeze' the cuts in the B and D regions used in the ABCD estimation for QCD background. For each selection, the B/D values diverge 
from the value calculated at the beginning of the selection when the statistics in the B and/or D region drops significantly. To avoid large statistical errors in the normalization calculated for 
the multi-jet contribution in the A region, the yields from the B and D regions used in the ABCD calculation are frozen, i.e. no further cuts applied after the earliest cut in the selection which has
a B/D ratio consistent with the last statistically stable B/D ratio. Using this method, the multi-jet modelling is kept as close as possible to the phase space in the final signal region while taking
advantage of higher statistics earlier in the cutflow. All numbers in this study were conducted with a \ttbar normalization factor equal to 1.0, i.e. no data-driven normalization was applied to \ttbar. 

\begin{table}[h!]
\centering
\begin{tabular}{c|c|c|c}
\hline\hline
\multicolumn{4}{c}{QCD $B/D$ Values, Non-resonant Selection}\\\hline\hline
mww             & bbpt210               & bbpt300               & wwpt250          \\\hline
0.36 $\pm$ 0.01 	& 0.35 $\pm$ 0.07 	& 0.49 $\pm$ 0.30 	& 0.14 $\pm$ 0.10  \\\hline 

\hline\hline
\end{tabular}

\begin{tabular}{c|c|c|c}
\multicolumn{4}{c}{QCD $B/D$ Values, Low Mass (m700) Selection}\\\hline\hline
mww                     & bbpt210               & wwpt250               & hh700            \\\hline
0.36 $\pm$ 0.01 	& 0.35 $\pm$ 0.07 	& 0.04 $\pm$ 0.02 	& 0.11 $\pm$ 0.07  \\\hline 

\hline\hline
\end{tabular}

\begin{tabular}{c|c|c|c}
\multicolumn{4}{c}{QCD $B/D$ Values, High Mass Selection}\\\hline\hline
bbpt350                 & wwpt250               & drww15                & hh2000            \\\hline
0.29 $\pm$ 0.04 	& 0.28 $\pm$ 0.04 	& 0.22 $\pm$ 0.06 	& 1.24 $\pm$ 0.86	\\\hline 
\hline\hline
\end{tabular}


\caption{Values calculated for $B/D$ at each stage in the non-resonant,
  low mass, and high mass selections. The estimate of multi-jet
  contribution in the A region uses the $R$ value calculated after the
   selection described in the text.} \label{tab:bdValues}

\end{table}

The QCD and total background yields obtained in the \mbb control region without freezing the B and D regions are shown in Tables~\ref{tab:noFreeze_nonRes}, ~\ref{tab:noFreeze_lowMass}, and~\ref{tab:noFreeze_highMass}.

\begin{table}[h!]
\centering
\begin{tabular}{l|c|c|c|c}
\hline\hline
\multicolumn{5}{c}{Non-resonant Selection in \mbb Control Region, No B/D Freezing}\\\hline\hline
Sample  	& mww 	& bbpt210 	& bbpt300 	& wwpt250 	 \\\hline
QCD 	& 13310.5 $\pm$ 500.3 	& 250.2 $\pm$ 30.6 	& 24.6 $\pm$ 3.0 	& 54.8 $\pm$ 6.7 	\\\hline 
\hline
Background Sum 	& 43849.0$\pm$ 509.2 	& 1017.9$\pm$ 33.7 	& 192.8$\pm$ 7.1 	& 153.2$\pm$ 8.4 \\\hline 
\hline
Data 	& 43902.0 	& 1069.0 	& 206.0 	& 138.0 	\\\hline 

\hline\hline
\end{tabular}
\caption{QCD and total background yields for the non-resonant
  selection without freezing the selection cuts used in the B and D
  regions, i.e. the yields in the B and D region after each cut are
  used in the ABCD calculation up until the \mbb cut. Non-monotonic
  QCD yields are observed.}
\label{tab:noFreeze_nonRes}
\end{table}

\begin{table}[h!]
\centering
\begin{tabular}{l|c|c|c|c}
\hline\hline
\multicolumn{5}{c}{Low Mass (m700) Selection in \mbb Control Region, No B/D Freezing}\\\hline\hline
Sample  	& mww 	& bbpt210 	& wwpt250 	& hh700 	 \\\hline
QCD 	& 13310.5 $\pm$ 500.3 	& 250.2 $\pm$ 30.6 	& 585.3 $\pm$ 71.7 	& 54.8 $\pm$ 6.7 	\\\hline 
\hline
Background Sum 	& 43849.0$\pm$ 509.2 	& 1017.9$\pm$ 33.7 	& 843.7$\pm$ 72.1 	& 104.7$\pm$ 7.6 	\\\hline 
\hline
Data 	& 43902.0 	& 1069.0 	& 367.0 	& 89.0 	\\\hline 

\hline\hline
\end{tabular}
\caption{QCD and total background yields for the low mass selection without freezing the selection cuts used in the B and D regions, i.e. the yields in the B and D region after each cut are used in the ABCD calculation up until the \mbb cut.}
\label{tab:noFreeze_lowMass}
\end{table}


\begin{table}[h!]
\centering
\begin{tabular}{l|c|c|c|c}
\hline\hline
\multicolumn{5}{c}{High Mass Selection in \mbb Control Region, No B/D Freezing}\\\hline\hline
Sample  	& bbpt350 	& wwpt250 	& drww15 	& hh2000    \\\hline
QCD 	& 1538.7 $\pm$ 252.7 	& 1359.5 $\pm$ 75.9 	& 486.4 $\pm$ 27.1 	& 4.6 $\pm$ 0.3 	\\\hline 
\hline
Background Sum 	& 14719.1$\pm$ 258.9 	& 12463.5$\pm$ 91.8 	& 3671.3$\pm$ 38.5 	& 222.8$\pm$ 7.1 	\\\hline 
\hline
Data 	& 14862.0 	& 12450.0 	& 3761.0 	& 250.0 	\\\hline 

\hline\hline
\end{tabular}
\caption{QCD and total background yields for the low mass selection without freezing the selection cuts used in the B and D regions, i.e. the yields in the B and D region after each cut are used in the ABCD calculation up until the \mbb cut.}
\label{tab:noFreeze_highMass}
\end{table}

The B/D ratio at the last cut in each selection have large errors (near or larger than 100\%) and are found to be unstable. The yields after freezing the B and D regions to their yields after the earliest selection cut with a B/D ratio consistent with the last statistically stable B/D ratio (after \ptbb $>$ 210 GeV for the non-resonant and low-mass selections and after \ptww $>$ 250 GeV for the high mass selection) are shown in Tables~\ref{tab:freeze_nonRes}, ~\ref{tab:freeze_lowMass}, and~\ref{tab:freeze_highMass}.

\begin{table}[h!]
\centering
\begin{tabular}{l|c|c|c|c}
\hline\hline
\multicolumn{5}{c}{Non-resonant Selection in \mbb Control Region, B/D Frozen after \ptbb $>$ 210}\\\hline\hline

Sample  	& mww 	& bbpt210 	& bbpt300 	& wwpt250 	 \\\hline
QCD 	& 13310.5 $\pm$ 500.3 	& 250.2 $\pm$ 30.6 	& 33.7 $\pm$ 4.1 	& 21.4 $\pm$ 2.6 	\\\hline 
\hline
Background Sum 	& 43849.0$\pm$ 509.2 	& 1017.9$\pm$ 33.7 	& 201.9$\pm$ 7.6 	& 119.8$\pm$ 5.7 	\\\hline 
\hline
Data 	& 43902.0 	& 1069.0 	& 206.0 	& 138.0 	\\\hline

\hline\hline
\end{tabular}
\caption{QCD and total background yields for the non-resonant selection after freezing the selection cuts used in the B and D regions after requiring \ptbb $>$ 210 GeV. Monotonic QCD yields are now observed.}
\label{tab:freeze_nonRes}
\end{table}

\begin{table}[h!]
\centering
\begin{tabular}{l|c|c|c|c}
\hline\hline
\multicolumn{5}{c}{Low Mass (m700) Selection in \mbb Control Region, B/D Frozen after \ptbb $>$ 210}\\\hline\hline

Sample  	& mww 	& bbpt210 	& wwpt250 	& hh700 	 \\\hline
QCD 	& 13310.5 $\pm$ 500.3 	& 250.2 $\pm$ 30.6 	& 72.4 $\pm$ 8.9 	& 16.3 $\pm$ 2.0 	\\\hline 
\hline
Background Sum 	& 43849.0$\pm$ 509.2 	& 1017.9$\pm$ 33.7 	& 330.7$\pm$ 12.1 	& 66.2$\pm$ 4.1 \\\hline 
\hline
Data 	& 43902.0 	& 1069.0 	& 367.0 	& 89.0 	\\\hline

\hline\hline
\end{tabular}
\caption{QCD and total background yields for the low mass (m700) selection after freezing the selection cuts used in the B and D regions after requiring \ptbb $>$ 210 GeV.}
\label{tab:freeze_lowMass}
\end{table}


\begin{table}[h!]
\centering
\begin{tabular}{l|c|c|c|c}
\hline\hline
\multicolumn{5}{c}{High Mass Selection in \mbb Control Region, B/D Frozen after \ptww $>$ 250 GeV}\\\hline\hline
Sample  	& bbpt350 	& wwpt250 	& drww15 	& hh2000 	 \\\hline
QCD 	& 1538.7 $\pm$ 252.7 	& 1359.5 $\pm$ 75.9 	& 392.7 $\pm$ 21.9 	& 20.7 $\pm$ 1.2 	\\\hline 
\hline
Background Sum 	& 14719.1$\pm$ 258.9 	& 12463.5$\pm$ 91.8 	& 3577.5$\pm$ 35.0 	& 238.9$\pm$ 7.2 	\\\hline
\hline 
Data 	& 14862.0 	& 12450.0 	& 3761.0 	& 250.0 	\\\hline

\hline\hline
\end{tabular}
\caption{QCD and total background yields for the low mass selection after freezing the selection cuts used in the B and D regions after requring \ptww $>$ 250 GeV.}
\label{tab:freeze_highMass}
\end{table}

After freezing, the yields in the non-resonant \mbb control region are monotonically decreasing as expected, and the absolute statistical error on the QCD estimate in the final signal region is significantly reduced compared to the yields obtained without freezing the B and D regions.

\clearpage
\section{QCD Lepton Flavour Composition after Preselection Criteria for $\sigma_{d0}$ distribution}

We estimate the lepton flavour composition in the QCD sample and SM signal sample just after the preselection criteria.
By using the $\sigma_{d0}$ distribution we identify the lepton flavour and the origin of each lepton.
The majority of the muons results to be from bottom meson while most of the electrons come from photon conversion (about 70\%).
The impact of the $\sigma_{d0}$ cut on the two population is mostly independent from the lepton flavour, namely the cut
$|\sigma_{d0}| \le 2.0$ removes for muon leptons (76 $\pm$ 100)\% for QCD sample and (6.9 $\pm$ 1.1)\% for signal sample, while for the electons these fractions are (27.9 $\pm$ 5.0)\% for QCD sample and (5.7 $\pm$ 0.9)\% for signal.
The Table~\ref{tab:LepFlav} shows the complete lepton flavour composition.

\begin{table}[h!]
\centering
\begin{tabular}{|c|c|}
\hline\hline
\multicolumn{2}{c}{Electrons}\\\hline\hline
Photon Conversion  & 65.3\%  \\
Not Defined & 15.0\% \\
Bottom Meson & 11.9\% \\
Dalitz Decay &4.6\% \\
Others & $\le$ 0.1\% \\
\hline \hline
\multicolumn{2}{c}{Muons}\\\hline\hline
\hline \hline
Bottom Meson  & 99\% \\
Charm Barion  & 1\% \\
\hline\hline
\end{tabular}
\caption{Lepton flavour composition in QCD sample. The lepton origin is reported for each event passing the preselection criteria. }
\label{tab:LepFlav}
\end{table}

