%%%%Use this as a filler to get the template working
%%Introduction
\chapter{Simulation and Event Reconstruction}
\label{chap:EvtReco} 
\section{Simulation}
In order to draw conclusion from ATLAS data, it is necessary to compare to theoretical predictions. For particle collisions, it is not practical to create exact predictions, especially including detector effects such as resolution. To get the best estimate of these effects, ATLAS uses the Monte Carlo (MC) method to simulate data and detector response to the incident particles. This is done in multiple steps as illustrated by figure ~\ref{fig:eventsim}. These steps are the simulation of the hard process, where the deep inelastic collision simulated using the initial state (Parton Distribution Functions) and interaction amplitudes; the parton shower; the hadronization; the detector simulation; and finally the reconstruction. These steps together form the complete MC simulation of ATLAS data.\newline

\begin{figure}[h]
\begin{center}
\includegraphics*[width=0.70\textwidth] {figures/event_simulation}
\caption[Event generation cartoon]{Pictorial representation of how an event is generated \cite{Wanotayaroj:2242196}}
\label{fig:eventsim}
\end{center}
\end{figure}

\subsection{Parton Distribution Functions}
\indent At the energies at the LHC, collisions usually do not involve entire protons. Instead, they involve constituents known as partons. Protons, while often described as two up quarks and a down quark, contain a sea of gluons. This sea of gluons also creates many virtual quark-antiquark pairs. The valance quarks for the proton are up and down quarks. These valance quarks are the primary role players in lower energy inelastic interactions. At the LHC, the collision energies are sufficient for deep inelastic scattering, where the affects of the sea quarks and gluons are non-trivial. For di-Higgs events, the dominant form of production is gluon-gluon fusion (ggF). This internal structure of the proton is described  by a Parton Distribution Function (PDF), figure ~\ref{fig:pdf}. A PDF shows the probability density of finding a parton carrying a momentum fraction ${x}$ at a squared energy scale. %(http://www.scholarpedia.org/article/Introduction_to_Parton_Distribution_Functions)
\newline

\begin{figure}[h]
\begin{center}
\includegraphics*[width=0.65\textwidth] {figures/pdf.jpg}
\caption[Parton distribution function]{The bands are ${x}$ times the unpolarized parton distributions
${f(x)}$ (where ${f = u_{v}, d_{v}, \bar{u}, \bar{d}, s \simeq{} \bar{s}, c = \bar{c}, b = \bar{b}, g}$) obtained in NNLO NNPDF3.0
global analysis at scales ${\mu^{2} = 10  \text{ GeV}^{2}}$
(a) and ${\mu^{2} = 100  \text{ GeV}^{2}}$ (b), with
${\alpha_{s}(M^{2}_{Z}) = 0.118}$.}
\label{fig:pdf}
\end{center}
\end{figure}

\subsection{Hard Scattering}
\indent The hard scattering process can be described using Feynman diagrams. These diagrams are a pictorial representation of amplitudes. These amplitudes go into calculating the matrix elements (ME) of various interactions. This ME describes the probability of a certain interaction occurring. In the event generation, these MEs are calculated to a specified order in perturbation theory. Common examples are leading order(LO), next-to-leading order(NLO) , and so on. The higher the order of the calculation, the more accurate the predictions. However, higher orders can be extremely hard to theoretically calculate, often restricting the level of the event generator. \newline
\subsection{Parton Shower Calculation}
\indent  After the ME generator, the hard partons are used as the inputs to the Parton Shower (PS) calculation. A parton shower is the evolution from the quarks and gluons produced in the hard interaction to the final state hadrons and other particles seen in the detector through QCD processes. The PS calculation models this showering process. \newline
In an interaction, colored particles can spontaneously emit gluons. These gluons, in turn, create either more gluons, or quark-antiquark pairs. This can happen either before (ISR) or after (FSR) the hard scattering process. The PS generator can also describes the hadronization and subsequent decay of the hadrons into the final state particles. \newline
The precision of the PS generators are described similarly to the ME, with their contributions coming in as leading log (LL), next-to-leading-log (NLL), etc. for the parton showering process. \newline
\subsection{Detector Simulation}
\indent The MC simulation up to this point can be done with generators that are written outside of the ATLAS collaboration. These generators are used by ATLAS to simulate the underlying processes which are fed into the detector simulation software. ATLAS uses GEANT4 to handle this propagation\cite{geant4} through the detetor. GEANT4 uses a detailed geometric description of the ATLAS detector to simulate particle interactions with the detector material. This includes simulation of energy deposition and the readout process. \newline
\indent The final result of the MC event generation is a set of simulated data that resembles actual data from the p-p collisions in the ATLAS detector.
\subsection{Reconstruction}
Once the data has been simulated, it is necessary to transform it into meaningful objects through reconstruction. There are two main types of reconstruction in ATLAS: turning patterns of hits in tracking detectors (Inner Detector and Muon Spectrometer) into tracks with direction and momentum information, and turning energy deposits in the calorimeter into calibrated energy deposits. These objects are then used to build a picture of the physics event through particle identification and event reconstruction. 
\section{Particle Identification}
For all events, either MC or actual collision data, it is important to be able to identify and reconstruct the underlying physics event. In particle collisions, the energy from the final state particles is deposited in the various subdetectors within ATLAS. These energy deposits must be translated to physically meaningful objects. This is the task of the event reconstruction, to use the ATLAS detector to recreate the final state particles for any given interaction. For this analysis, the final state particles present in the signal events are a lepton, either an electron or a muon; a neutrino, in the from of missing transverse energy; two light flavor quarks; and two b quarks. Each of these particles has a particular signal in each of the subdetectors, figure ~\ref{fig:crossSec}.

\begin{figure}[h]
\begin{center}
\includegraphics*[width=0.70\textwidth] {figures/layers}
\caption[Cartoon showing particle interating in ATLAS detector subsystems]{Event Cross Section in a computer generated image of the ATLAS detector \cite{Pequenao:1096081}}
\label{fig:crossSec}
\end{center}
\end{figure}


\subsection{Electrons}
Electrons are reconstructed by fitting a track using the Inner Detector and matching this track to an energy cluster in the EM calorimeter\cite{Tarna:2286383}. As an electron passes through the EM calorimeter, it produces Bremsstahlung radiation photons. These photons then convert back to electron-positron pairs and the process repeats. This shower of electrons, positrons, and photons give the signature energy cluster in the calorimeter. Particles with the required Inner Detector track and matching EM energy cluster are selected as electron candidates.\newline
\indent Electron identification algorithms are applied to these electron candidates. These algorithms separate prompt, isolated electron candidates from backgrounds such as converted photons and misidentified jets. The electron identification algorithm uses the energy-momentum ratio, shower shape, track and track-to-cluster matching to identify electron candidates, with E/p being the most important discriminant. There are three identification working points for electron identification: Loose, Medium, and Tight. The operating points with higher background rejection are a subset of electron candidates with lower background rejection with a tighter background rejection giving a lower electron efficiency. \newline
\indent The isolation variables quantify the energy around the electron candidate and allow us to disentangle prompt electrons 
from other, non-isolated electron candidates such as electrons originating 
from converted photons produced in hadron decays, electrons from heavy flavor hadron decays, 
and light hadrons mis-identified as electrons. The isolation variable we use for reconstructed electrons 
is \textit{Track}-based isolation, ${p_{T}^{\mathrm{varcone0.2}}}$, defined as the sum of transverse momenta
of all tracks, satisfying quality requirements, within a cone around the electron candidate of ${\Delta R = 0.2}$ or of ${10\text{ GeV}/E_{T}}$ for high energy electrons, where $E_{T}$ is the transverse energy of the electron candidate.

A more detailed discussion on the electron likelihood identification and isolation variables 
and their performance with Run 2 data can be found in Ref.~\cite{ATLAS-CONF-2016-024}. 
The electron energy scale is calibrated such that it is uniform throughout the detector and the residual differences
between data and simulation are corrected. The calibration strategy is based on the same strategy developed 
in Run 1 ~\cite{ATLAS-EGAMMACALIB-RUN1} and updates to the calibration strategy for Run 2 is 
documented in Ref.~\cite{ATL-PHYS-PUB-2016-015}.
\subsection{Muons}
The Muon Spectrometer specializes in muon detection and precision momentum measurement. Unsurprisingly, this makes the Muon Spectrometer (MS) a vital part of muon identification, but it is not the only subdetector used. The Inner Detector is also has an important part in reconstructing muons. In ATLAS, muon reconstruction is performed independently in the Inner Detector and the MS. The information is then combined to form muon tracks. In the Inner Detector, the muons are reconstructed similarly to any other charged particle.\newline
\indent In the MS, the reconstruction looks for a hit pattern within each chamber to form segments \cite{Aad:2016jkr}. The MDT segments are combined using a straight-line fit within a single layer. Segments in the CSCs are combined using a combinatorial search in the ${\eta}$ and ${\phi}$ planes. \newline
\indent Muon candidates are built by fitting together hits from segments in different layers. A combinatorial search, using segments in the middle layer as seeds, is performed. The inner and outer layers are then used as seeds as the search is extended. A minimum of 2 segments are required to build a track. It is possible for a segment to be included in multiple tracks, an overlap removal algorithm selects the best assigned track or can allow for a segment to be shared between two tracks. A global ${\chi^{2}}$ fit is performed on the hits of each track. If the ${\chi^{2}}$ of the fit passes a selection criteria, the track is accepted.\newline
\indent The information from the Inner Detector and the MS are then combined to give a muon signature. The combination method depends on the information available. The main method used is the Combined Muon reconstruction, where track reconstruction is performed in the Inner Detector and MS independently. Most of these muons are reconstructed using an ''outside-in" reconstruction. This  means tracks in the MS are extrapolated inward and matched to an Inner Detector track. \newline
The muon isolation variables are similar to the electron isolation variables above 
which is the \textit{track}-based isolation, ${p_{T}^{\mathrm{varcone0.3}}}$, defined as the 
sum of transverse momenta of all tracks, satisfying quality requirements, 
within a cone of $\Delta R = \text{ min}(0.3,10 \GeV/\pt)$
around the candidate muon.


The performance of the muon identification and isolation variables are documented in Ref.~\cite{Aad:2016jkr}.
%The identification and isolation efficiencies for muons are corrected using scale factors derived 
%using $Z\to \mu \mu$ and $J/\Psi \to \mu \mu$ events\footnote{The scale factors are provided by the
%MuonEfficiencyScaleFactors tool, as outlined in 
%\href{https://twiki.cern.ch/twiki/bin/view/AtlasProtected/MCPAnalysisGuidelinesMC15}{MCPAnalysisGuidelinesMC15 twiki}}.

Corrections to the muon momentum scale and resolution are applied to MC simulation using the MuonCalibrationAndSmearingTool\footnote{as prescribed 
in \href{https://twiki.cern.ch/twiki/bin/view/AtlasProtected/MCPAnalysisGuidelinesMC15\#Muon_momentum_scale_and_resoluti} {MCPAnalysisGuidelinesMC15 twiki}} 
to correct for  data/MC differences. The correction factors were derived from data/MC 
simulation comparisons with $Z\to \mu \mu$ and $J/\Psi \to \mu \mu$ events (the calibration procedure to derive the factors 
is documented in Ref.~\cite{Aad:2016jkr}).
\subsection{Jets}
Quarks very quickly undergo showering. Once the average energy of a quark (or gluon) reaches ~1 GeV, the particles hadronize, with only the top quark decaying before hadronizing. If we could measure every hadron and correctly assign them to the underlying quarks, energy and momentum conservation would allow the exact momentum and energy of the quark could be determined. However, since this is not possible, another method must be used to try and make this association. Instead, collections of hadrons are formed and these are what deposit energy in the ATLAS detector. These collections of energy are called jets and can be made from various detector object. In this analysis in particular, two different types of jets are used: calo-jets, jets constructed from energy deposited in the calorimeters; and track-jets, jets constructed from tracks in the Inner Detector. \newline
\indent Since a jet is not a physical object, rather a collection of energy-momentum 4-vectors, there are many ways to define a jet. A jet algorithm takes a set of input 4-vectors and combines them into one or more jet objects based upon some criteria for separating and grouping inputs. Jets can be made with either energy deposits (calo-jets) or tracks (track-jets). The process of creating calo-jets is described here but the method for making track-jets is similar.  Two important characteristics of any jet algorithm are Infrared (IR) Safety and Collinear (CL) Safety. For a jet algorithm to be IR Safe, the addition or subtraction of small energy deposits will not change the jet collection. A jet algorithm is CL Safe if splitting or merging high transverse momentum particles does not change the jet collection. Figure \ref{fig:IR_CL} illustrates both IR and CL Safety.

\begin{figure}[h]
\begin{center}
\includegraphics*[width=0.70\textwidth] {figures/IR_CL_safe}
\caption[Illustration of Infrared and collinear safety]{Illustration of the infrared sensitivity of a cursory designed jet algorithm (top). Illustration of the product of a collinear unsafe jet algorithm. A collinear splitting changes the number of jets (bottom). \cite{Isildak:2013kfa}.}
\label{fig:IR_CL}
\end{center}
\end{figure}

\indent Some examples of jet algorithms are visualized in figure ~\ref{fig:jetalgo}. For this analysis, the ${\text{anti-}k_{t}}$ algorithm is selected. In addition to being IR and CL safe, the ${\text{anti-}k_{t}}$ algorithm gives roughly circular jets. This makes calculating the energy density much easier than non-circular jets. The ${\text{anti-}k_{t}}$ algorithm calculates the distance between objects $i$ and $j$ and $i$ and the beam $B$. If ${d_{ij}}$ is smaller than ${d_{iB}}$, the objects are combined. If ${d_{iB}}$ is smaller the object is removed and the algorithm is rerun. An important distinction between ${\textrm{anti-}k_{t}}$ and other jet algorithms is the definition of the distances ${d_{ij}\text{ and } d_{iB}}$
\begin{equation}
\begin{split}
d_{ij} = \text{ min}(k^{2p}_{ti},k^{2p}_{tj})\frac{\Delta^{2}_{ij}}{R^{2}},\\
d_{iB} = k^{2p}_{ti}
\end{split}
\end{equation}
Where ${k_{ti}}$ is the transverse momentum, ${\Delta}$ is the distance between objects, and ${p=-1}$. The ${\text{ anti-}k_{t}}$ algorithm has a radius parameter R. R acts as a cutoff radius for energy clustering and is not strictly a radius, as objects with a ${\Delta > R}$ can still be clustered together. The track-jets used in the analysis have R = 0.2, while the R = 0.4 (small-R) and R = 1.0 (large-R) calo-jets are used. 
\subsubsection{Large-R jets}
For decays with a high momentum to rest-mass ratio, such as the $W\rightarrow qq$ decay, it is impossible to separate energies cleanly into jets with R = 0.4. Instead, to measure the energy/momentum of the $W$ it is advantageous to use a larger radius paramenter. The large-R jets are clustered using the ${\mathrm{anti-}k_{t}}$ jet algorithm \cite{antikt_algorithm} with topological calorimeter clusters as inputs. The clusters are calibrated to the ``local hadronic cell weighting"(LCW) scale \cite{ATLAS-TopoClustering}.
In order to minimize the effects from pileup on the large-R jet kinematics, the large-R jet is then groomed
using the trimming algorithm. The trimming algorithm removes subjets if the ratio of the subjet \pt over the large-R jet \pt is below some threshold\cite{Krohn:2009th}. This removes energy from pileup that is contained in the jet. The large-R jet energy and mass is then calibrated to the particle-level
scale. The calibration factors were derived from MC simulation of multijet events \cite{ATLAS-CONF-2016-035}.\newline
\subsubsection{Track jets}
Track jets are built by clustering Inner Detector tracks using the ${\text{anti-}k_{t}}$ algorithm with a radius parameter R = 0.2. The selected tracks are required to have \pt greater than 400 MeV and pass a loose set of cuts, as listed in reference \cite{ATL-PHYS-PUB-2015-035}. The smaller R parameter coupled with the fact that tracks have better angular resolution than calorimeter clusters, mean that the decay products of highly boosted heavy objects can still be resolved. The selected track jets are then associated to the large-R calorimeter jets via ghost association \cite{Cacciari:2008gn} method. A b-tagging algorithm is used to identify track jets which are likely to contain b-hadrons which consist of the b-quarks from the Higgs boson decay. The MV2c10 algorithm exploit the relatively long lifetime of B-hadrons with respect to lighter hadrons, as well as the kinematics of the charged particle tracks.
\subsubsection{Small-R jets}
Small-R jets are reconstructed from three-dimensional topological calorimeter 
clusters~\cite{ATLAS-TopoClustering} using the anti-$k_t$ jet 
algorithm~~\cite{antikt_algorithm} with a radius parameter of 0.4. This is the standard jet used in most ATLAS analyses.
Jet energies are corrected~\cite{ATLAS-JES-RUN2} for detector inhomogeneities, the non-compensating nature of the calorimeter, and the impact of multiple overlapping $pp$ interactions. Correction factors are derived using test beam, cosmic ray, $pp$ collision data, and a detailed Geant4 detector simulation.
Jet cleaning is applied to remove events with jets built from noisy
calorimeter cells or non-collision backgrounds, requiring that jets
are not of ``bad'' quality.\footnote{\textit{LooseBad} jets, 
defined on the \href{https://twiki.cern.ch/twiki/bin/view/AtlasProtected/HowToCleanJets2016}{HowToCleanJets2016 twiki}, 
are removed.}

To avoid selecting jets originating from pile-up interactions a ``jet vertex tagger'' (JVT) criterion~\cite{ATLAS-JVTPaper} is applied for jets with $\pt < 60$ GeV and $|\eta|< 2.5$ requiring a JVT $ > 0.59$ cut. This cut corresponds to the \texttt{\textbf{Default}} working point, as described on the \href{https://twiki.cern.ch/twiki/bin/view/AtlasProtected/JVTCalibration}{JVTCalibration twiki}.



\begin{figure}[h]
\begin{center}
\includegraphics*[width=0.40\textwidth] {figures/kt_stuff/kt}
\includegraphics*[width=0.40\textwidth] {figures/kt_stuff/ca}\\
\includegraphics*[width=0.40\textwidth] {figures/kt_stuff/siscone}
\includegraphics*[width=0.40\textwidth] {figures/kt_stuff/antikt}
\caption[A sample parton-level event]{A sample parton-level event, together with many random soft
``ghosts", clustered with four different jets algorithms, illustrating the ``active" catchment areas of
the resulting hard jets\cite{Cacciari:2008gp}.}
\label{fig:jetalgo}
\end{center}
\end{figure}

\subsection{b Tagging}\label{ssec:btag}
B-Hadrons, composite particles that contain b quarks, travel a small distance before they decay. This means the track from the decay products can be traced back to the point at which the b-hadron decays, called the secondary, or displaced, vertex, figure ~\ref{fig:bjets} . This displaced vertex is used to tag jets that are likely to come from b quarks through b-tagging algorithms.\newline
\indent In this analysis, the MV2c10 is used to tag b-jets \cite{ATL-PHYS-PUB-2016-012}. MV2 is a multivariate discriminant that combines 3 b-tagging algorithms. The c10 signifies a 10\% c-jet fraction in the background training sample. The three algorithms that are used as inputs to the MV2 discriminant are: an impact parameter-based algorithm, an inclusive secondary vertex reconstruction algorithm, and a decay chain multi-vertex reconstruction algorithm. For this analysis, the 85\% fixed-cut working point is used for b-jet identification.\newline
\indent The difference in the efficiency of $b$-tagging between data and simulation 
is taken into account by applying scale factors provided by the Flavour Tagging CP group, 
as prescribed on the \href{https://twiki.cern.ch/twiki/bin/view/AtlasProtected/BTagCalib2015}{BTagCalib2015 twiki}. 
The uncertainties associated with $b$-tagging are considered for $b$-, $c$- and light-flavor-induced jets, separately. 

\begin{figure}[h]
\begin{center}
\includegraphics*[width=0.70\textwidth] {figures/bjet}
\caption[Schmatic view of the tracks in a b-jet]{Schmatic view of the tracks in a b-jet \cite{HanssonAdrian:1397942}.}
\label{fig:bjets}
\end{center}
\end{figure}

\subsection{Missing Transverse Momentum}
Neutrinos do not interact with the detector as they pass through. This means they cannot be measured like the other particles. In order to measure neutrinos, ATLAS relies on the conservation of momentum. As previously mentioned, the exact collision energy is unknown, as each partons does not carry a consistent fraction of the proton energy. However, in the transverse plane, the plane perpendicular to the beam line, the total momentum is known to be very small. Before the collision, there is very little momentum in the transverse plane, on the order of 1 GeV. After the collision, this must also be true. This implies the vector summation of all objects should have approximately zero momentum in the transverse plane. Any imbalance in this is referred to as  Missing Transverse Momentum (\met). The energy symbol is used, however, we really mean vector momentum. The \met{} is constructed as the negative vector sum of all reconstructed objects with an additional soft term reconstructed from detector signal objects not associated with any object\cite{ATL-PHYS-PUB-2015-027}. 
\begin{equation}
E^{\mathrm{miss}}_{x(y)} = E^{\mathrm{miss, e}}_{x(y)}+E^{\mathrm{miss, \gamma}}_{x(y)} + E^{\mathrm{miss, \tau}}_{x(y)} + E^{\mathrm{miss, jets}}_{x(y)} + E^{\mathrm{miss, \mu}}_{x(y)} + E^{\mathrm{miss, soft}}_{x(y)}
\end{equation}
From the $x$ and $y$ components of ${E^{\mathrm{miss}}}$, the magnitude and azimuthal angle  are calculated.
\begin{equation}
\begin{split}
\met = \sqrt{(E^{\mathrm{miss}}_{x})^{2} + (E^{\mathrm{miss}}_{y})^{2}},\\
\phi^{\mathrm{miss}} = \arctan(E^{\mathrm{miss}}_{y}/E^{\mathrm{miss}}_{x})
\end{split}
\end{equation}
In this analysis, the \met is reconstructed using VHLooseElectrons, VHLooseMuons the analysis jets, and the track-based soft term.\newline
\indent The \met{} vector is a vector in the transverse plane, meaning it does not directly correspond to a neutrino. Additional information is needed to exactly reconstruct a neutrino. In this analysis, a Higgs mass constraint is used to supply the direction of the signal neutrino. \newline

