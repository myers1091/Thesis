%%%%Use this as a filler to get the template working
%%Introduction
\chapter{DiHiggs Production}

\section{Standard Model}
The Higgs potential in the SM contains the term:
\begin{equation}
\lambda(|\Phi^{\dagger}\Phi|)^{2}
\end{equation}
Expanding ${\Phi}$ around the VEV gives:
%https://arxiv.org/pdf/hep-ph/0403303.pdf
\begin{equation}
\phi = \frac{1}{\sqrt{2}}\binom{0}{\nu + H}
\end{equation}
Where ${\nu}$ is the VEV and ${H}$ is the Higgs field
This gives rise to the higher order Higgs self coupling terms of the potential.
\begin{equation}
V_{H} = \frac{1}{2}(2\lambda\nu^{2})H^{2} + \lambda\nu H^{3} + \frac{\lambda}{4}H^{4}
\end{equation}
The coefficient of the cubic term, ${\lambda\nu}$, is the trilinear Higgs self-coupling (${\lambda_{HHH}}$)\cite{Belusevic:2004pz}. This coupling can be probed at the LHC by measuring the cross section of events with two Higgs Bosons in the final state.\linebreak

\begin{figure}[h]
\begin{center}
\begin{tikzpicture}
\begin{feynman}
\vertex (i1);
\vertex [right = of i1] (t1);
\vertex [dot][below right =of t1] (t3);
\vertex [below left= of t3] (t2);
\vertex [left = of t2] (i2);
\vertex [right =of t3][dot](h){};
\vertex [above right = of h] (f1){\(H\)};
\vertex [below right = of h] (f2){\(H\)};
\diagram* {
(i1) -- [gluon] (t1),
(i2) -- [gluon] (t2),
(t1) -- [fermion](t3) -- [fermion] (t2) -- [fermion](t1),
(t3) -- [scalar,edge label'=\(H\)] (h),
(f1) -- [scalar] (h) -- [scalar](f2)
};
\vertex [right=0.75cm of h] {\(\lambda_{HHH}\)};
\end{feynman}
\end{tikzpicture}
\hspace{1cm}
\begin{tikzpicture}
\begin{feynman}
\vertex (i1);
\vertex [ right = of i1] (t1);
\vertex [right = of t1] (t2);
\vertex [below = of t2] (t3);
\vertex [left = of t3] (t4);
\vertex [ left=of t4] (i2);
\vertex [ right =of t2] (f1){\(H\)};
\vertex [ right=of t3] (f2){\(H\)};
\diagram* {
(i1) -- [gluon] (t1),
(i2) -- [gluon] (t4),
(t1) -- [fermion](t2) -- [fermion] (t3) -- [fermion](t4) -- [fermion] (t1),
(t2) -- [scalar] (f1),
(t3) -- [scalar] (f2)
};
\end{feynman}
\end{tikzpicture}
\caption{The dominate production method for double Higgs events at the LHC with ${\sqrt{s} = 13 TeV}$, with the triple Higgs coupling on the left}
\end{center}
\end{figure}
\label{fig:dihiggs}

\indent  Currently at the LHC, there are two dominate diagrams to produce diHiggs events, the trilinear Higgs coupling diagram and a box diagram, figure ~\ref{fig:dihiggs}. These diagrams interfere destructively, resulting in a low production cross section. This makes the trilinear coupling extremely hard to measure at the LHC but still accessible at the HL-LHC. However, new physics could manifest itself in an increase to the diHiggs production. Meaning any enhancement to the diHiggs cross section is indicative of physics beyond the Standard Model (BSM).




\section{Resonant Production}
There are several BSM models that may enhance the rate of diHiggs production. This section will give an overview of a few of these models
%Here will be a couple different theories
\subsection{Complex Higgs Singlet}
The addition of a complex scalar singlet to the SM results in three neutral scalar particles after spontaneous symmetry breaking, which mix to give mass eigenstates, including the observed 125 GeV scalar \cite{PhysRevD.97.015022}.\linebreak
\indent After spontaneous symmetry breaking, the fields are defined as:
\begin{equation}
\Phi = \binom{0}{\frac{h+\nu}{\sqrt{2}}}; S_{c} = \frac{1}{\sqrt{2}}(S+\nu_{s} + i(A + \nu_{A})).
\end{equation}
The normalizable scalar potential is
\begin{equation}
\begin{split}
V(\Phi,S_{c}) = \frac{\mu^{2}}{2}\Phi^{\dagger}\Phi + \frac{\lambda}{4}(\Phi^{\dagger}\Phi)^{2} \\
+ (\frac{1}{4}\delta_{1}\Phi^{\dagger}\Phi{}S_{c} + \frac{1}{4}\delta_{3}\Phi^{\dagger}\Phi{}S_{c}^{2} \\
+ a_{1}S_{c} + \frac{1}{4}b_{1}S_{c}^{2} + \frac{1}{6}e_{1}S_{c}^{3} + \frac{1}{6}e_{2}S_{c}|S_{c}|^{2} \\
+ \frac{1}{8}d_{1}S^{4} + \frac{1}{8}d_{3}S_{c}^{2}|S_{c}|^{2} + H.C.) \\
+ \frac{1}{4}d_{2}(|S_{2}|^{2})^{2} + \frac{\delta_{2}}{2}\Phi^{\dagger}\Phi|S_{c}|^{2} + \frac{1}{2}b_{2}|S_{c}|^{2}
\end{split}
\end{equation}
The mass eigenstate fields are given by:
\begin{equation}
\begin{pmatrix}
h_{1}\\
h_{2}\\
h_{3}
\end{pmatrix}
= \begin{pmatrix}
c_{1} & -s_{1} & 0\\
s_{1}c_{2} & c_{1}c_{2} & s_{2}\\
s_{1}s_{2} & c_{1}s_{2} & -c_{2}
\end{pmatrix} 
\begin{pmatrix}
h\\
S\\
A\\
\end{pmatrix}
\end{equation}
where ${c_{i} = \cos{\theta_{i}}}$.
Assigning the SM-like Higgs boson as ${h_{1}}$ with ${m_{1} = 125 GeV and \nu = 246 GeV}$,  the coupling of ${h_{1}}$ to SM particles is dominate, suppressed by a factor of ${c_{1}}$ from the SM rate, with the ${h_{2}}$ couplings suppressed by ${s_{1}c_{2}}$ and ${h_{3}}$ couplings suppressed by ${s_{1}s_{2}}$. ATLAS has sent limits of ${c_{1} > 0.94}$ at 95\% CL in RUN-1. As the ${h_{1}}$ couplings become more SM-like (${\theta_{1}\rightarrow{0}}$), the allowed ${h_{2}}$ couplings become suppressed.\linebreak

\begin{figure}[h]
\begin{center}
\begin{tikzpicture}
\begin{feynman}
\vertex (i1);
\vertex [right = of i1] (t1);
\vertex [dot][below right =of t1] (t3);
\vertex [below left= of t3] (t2);
\vertex [left = of t2] (i2);
\vertex [right =of t3][dot](h){};
\vertex [above right = of h] (f1){\(h_{j}\)};
\vertex [below right = of h] (f2){\(h_{k}\)};
\diagram* {
(i1) -- [gluon] (t1),
(i2) -- [gluon] (t2),
(t1) -- [fermion](t3) -- [fermion] (t2) -- [fermion](t1),
(t3) -- [scalar,edge label'=\(h_{i}\)] (h),
(f1) -- [scalar] (h) -- [scalar](f2)
};
\end{feynman}
\end{tikzpicture}
\hspace{1cm}
\begin{tikzpicture}
\begin{feynman}
\vertex (i1);
\vertex [ right = of i1] (t1);
\vertex [right = of t1] (t2);
\vertex [below = of t2] (t3);
\vertex [left = of t3] (t4);
\vertex [ left=of t4] (i2);
\vertex [ right =of t2] (f1){\(h_{j}\)};
\vertex [ right=of t3] (f2){\(h_{k}\)};
\diagram* {
(i1) -- [gluon] (t1),
(i2) -- [gluon] (t4),
(t1) -- [fermion](t2) -- [fermion] (t3) -- [fermion](t4) -- [fermion] (t1),
(t2) -- [scalar] (f1),
(t3) -- [scalar] (f2)
};
\end{feynman}
\end{tikzpicture}
\caption{Feynman diagrams for the production of ${h_{j}h_{k}}$, ${i, j, k = 1, 2, 3}$.}
\label{fig:FeyComp}
\end{center}
\end{figure}


\indent In the limit of ${\theta_{2}\rightarrow{0}}$, which is in agreement with the single Higgs rates, ${h_{3}}$ does not directly couple to SM particles and can only be observed in diHiggs production with the largest rate from ${gg\rightarrow h_{2}\rightarrow h_{1}h_{3}}$, figure ~\ref{fig:FeyComp} in paper. For a range of masses ${m_{2} and m_{3}}$ the rate of production of ${h_{1}h_{3} >> h_{1}h_{1}}$, figure ~\ref{fig:CSH6}. \linebreak

\begin{figure}[h]
\begin{center}
\includegraphics[scale=0.65]{figures/CompHiggsSing_Fig6}
\caption{Regions of parameter space allowed by limits on oblique parameters, perturbative unitarity and the minimization of the potential where the rate for ${h_{1}h_{3}}$ production is significantly larger than the SM ${h_{1}h_{1}}$ rate at ${\sqrt{S} = 13 TeV}$.}
\label{fig:CSH6}
\end{center}
\end{figure}


\indent The enhancement can be see in the potential where 
\begin{equation}
V \rightarrow \frac{1}{2}\lambda_{211}h_{1}^{2}h_{2} + \frac{1}{2}\lambda_{311}h_{1}^{2}h_{3} + \frac{1}{2}\lambda_{331}h_{1}h_{3}^{2} + \frac{1}{2}\lambda_{321}h_{1}h_{2}h_{3} + . . .
\end{equation}
So while the SM trilinear Higgs coupling is determined by ${m_{h}}$, with this extension, the coupling is much less constrained. This leads to enhanced values seen in figure \ref{fig:CHS8}.
\begin{figure}[h]
\begin{center}
\includegraphics[scale=0.65]{figures/CompHiggsSing_Fig8}
\caption{Region of parameter space allowed by limits on oblique parameters, perturbative unitarity and the minimization of the potential where the ${h_{1}h_{1}h_{1}}$ trilinear coupling is greater than 5 times the SM value.}
\label{fig:CHS8}
\end{center}
\end{figure}

\subsection{Real Higgs Singlet Extension}
One simple explanation of an enhanced diHiggs production rate at the LHC is the addition of a real scalar Higgs singlet, S \cite{Lewis:2017dme}. In this model, S can only interact with the SM through the Higgs field. In the case where there is no ${Z_{2}}$ symmetry, the scalar field S mixes with the SM Higgs boson. If the mass is large enough, it is possible for S to decay to two on-shell SM Higgs Bosons, significantly enhancing the diHiggs production rate.\linebreak
\indent The most general potential that can be added is 
\begin{equation}
V(H,S) = -\mu^{2}H^{\dagger}H + \lambda(H^{\dagger}H)^{2} + \frac{a_{1}}{2}H^{\dagger}HS + \frac{a_{2}}{2}H^{\dagger}HS + b_{1}S + \frac{b_{2}}{2}S^{2} + \frac{b_{3}}{3}S^{3} + \frac{b_{4}}{4}S^{4}.
\end{equation}
Where H is ${\phi_{0} = \frac{(h + \nu)}{\sqrt{2}}}$ and ${<\phi_{0}> = \frac{\nu}{2}}$, while ${S = s + x}$ where ${x}$ is the vev of S. By shifting the field, it is possible the set ${x = 0}$. After electroweak symmetry breaking the fields mix to give the two mass eigenstates
\begin{equation}
\binom{h_{1}}{h_{2}} = 
\begin{pmatrix}
\cos{\theta} & \sin{\theta}\\
-\sin{theta} & \cos{\theta}
\end{pmatrix}
\binom{h}{s}
\end{equation}
With ${m_{1} = 125 GeV}$, the free parameters are ${m_{2}, \theta,a_{2},b_{2}}$ and ${b_{4}}$. For diHiggs production, in the case of ${m_{2}>2m_{1}}$, the important piece of the potential is 
\begin{equation}
V(h_{1}h_{2}) \supset \frac{\lambda_{111}}{3!}h_{1}^{3} + \frac{\lambda_{211}}{3!}h_{2}h_{1}^{2}
\end{equation}
This give an additional diagram in the diHiggs production, figure ~\ref{fig:FeyRes}, for ${250 GeV \leq m_{2}}$.
\begin{figure}[h]
\begin{center}
\begin{tikzpicture}
\begin{feynman}
\vertex (i1);
\vertex [right = of i1] (t1);
\vertex [dot][below right =of t1] (t3);
\vertex [below left= of t3] (t2);
\vertex [left = of t2] (i2);
\vertex [right =of t3](h){};
\vertex [above right = of h] (f1){\(h_{1}\)};
\vertex [below right = of h] (f2){\(h_{1}\)};
\diagram* {
(i1) -- [gluon] (t1),
(i2) -- [gluon] (t2),
(t1) -- [fermion](t3) -- [fermion] (t2) -- [fermion](t1),
(t3) -- [scalar,edge label'=\(h_{i}\)] (h),
(f1) -- [scalar] (h) -- [scalar](f2)
};
\end{feynman}
\end{tikzpicture}
\caption{Feynman diagram for ${h_{2}\rightarrow h_{1}h_{1}}$.}
\label{fig:FeyRes}
\end{center}
\end{figure}

\indent Varying the values of ${b_{4}}$ and ${\sin^{2}{\theta}}$, it is found that the maximum branching ratio (BR) for ${h_{2}\rightarrow h_{1}h_{1}}$ if obtained with ${b_{4} = 4.2, \sin^{2}\theta = 0.12}$. Figure XXX, shows the minimum and maximum BR as a function of ${m_{2}}$. The largest BR is when ${m \approx 280 GeV}$ of ${BR(h_{2}\rightarrow h_{1}h_{1} = 0.76}$. This corresponds to an enhancement in diHiggs production of approximately 30 times the SM cross section.

\begin{figure}[h]
\begin{center}
\includegraphics[scale=0.5]{figures/Ian6}
\caption{Maximum and minimum allowed ${BR(h_{2}\rightarrow h_{1}h_{1})}$ as a function of ${m_{2}}$ for ${b_{4} = 4.2}$ and ${\sin^{2}{\theta} = 0.12}$.}
\label{fig:Ian6}
\end{center}
\end{figure}
