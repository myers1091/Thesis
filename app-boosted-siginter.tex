\begin{figure}[!h]
\begin{center}
\includegraphics*[width=0.49\textwidth]{./figures/boosted/LimitsUnblinded/Limits_RebinWSMaker_Baseline_ForNote_UNBLINDED_7000XXXX_GravitonC20}
\includegraphics*[width=0.49\textwidth]{./figures/boosted/LimitsUnblinded/Limits_RebinWSMaker_Baseline_ForNote_UNBLINDED_7000XXXX_GravitonC10RW}
\caption{Expected and observed upper limits at 95\% C.L. on the cross-section of resonant Higgs boson pair production
as a function of the signal Graviton c=2.0 (left) and c=1.0 (right) resonance mass.}
\label{fig:boosted_siginter_limit_obs_gravMC}
\end{center}
\end{figure}
\FloatBarrier

\subsection{Acceptance interpolation}
\label{app:boosted_siginter_acceptance}

\begin{figure}[!h]
\begin{center}
\includegraphics*[width=0.45\textwidth]{./figures/boosted/SigInterpolation/AcceptanceRatio_C10_Fit.pdf}
\includegraphics*[width=0.45\textwidth]{./figures/boosted/SigInterpolation/AcceptanceRatio_C20_Fit.pdf}
\caption{Ratio of acceptance.}
\label{fig:boosted_siginter_acceptanceratio}
\end{center}
\end{figure}
\FloatBarrier


\subsection{$m_{hh}$ shape interpolation}
\label{app:boosted_siginter_shape}

\begin{figure}[!h]
\begin{center}
\includegraphics*[width=0.45\textwidth]{./figures/boosted/SigInterpolation/GravitonFit_Ghh1000C20_Bukin}
\includegraphics*[width=0.45\textwidth]{./figures/boosted/SigInterpolation/GravitonFit_Ghh1500C20_Bukin}\\
\includegraphics*[width=0.45\textwidth]{./figures/boosted/SigInterpolation/GravitonFit_Ghh2000C20_Bukin}
\includegraphics*[width=0.45\textwidth]{./figures/boosted/SigInterpolation/GravitonFit_Ghh3000C20_Bukin}
\caption{Bukin functional fit to $m_{hh}$ distribution for several mass points of the graviton c=2.0 signal model.}
\label{fig:boosted_siginter_bukinfit_c20}
\end{center}
\end{figure}
\FloatBarrier

\begin{figure}[!h]
\begin{center}
\includegraphics*[width=0.45\textwidth]{./figures/boosted/SigInterpolation/GravitonFit_Ghh1000C10RW_Bukin}
\includegraphics*[width=0.45\textwidth]{./figures/boosted/SigInterpolation/GravitonFit_Ghh1500C10RW_Bukin}\\
\includegraphics*[width=0.45\textwidth]{./figures/boosted/SigInterpolation/GravitonFit_Ghh2000C10RW_Bukin}
\includegraphics*[width=0.45\textwidth]{./figures/boosted/SigInterpolation/GravitonFit_Ghh3000C10RW_Bukin}
\caption{Bukin functional fit to $m_{hh}$ distribution for several mass points of the graviton c=1.0 signal model.}
\label{fig:boosted_siginter_bukinfit_c10RW}
\end{center}
\end{figure}
\FloatBarrier

\begin{figure}[!h]
\begin{center}
\includegraphics*[width=0.45\textwidth]{./figures/boosted/SigInterpolation/ParamFit_GhhC20_Bukin_Graviton_Para0_max}
\includegraphics*[width=0.45\textwidth]{./figures/boosted/SigInterpolation/ParamFit_GhhC20_Bukin_Graviton_Para1_x0}\\
\includegraphics*[width=0.45\textwidth]{./figures/boosted/SigInterpolation/ParamFit_GhhC20_Bukin_Graviton_Para2_sigma}
\includegraphics*[width=0.45\textwidth]{./figures/boosted/SigInterpolation/ParamFit_GhhC20_Bukin_Graviton_Para3_xi}\\
\includegraphics*[width=0.45\textwidth]{./figures/boosted/SigInterpolation/ParamFit_GhhC20_Bukin_Graviton_Para4_rhoL}
\includegraphics*[width=0.45\textwidth]{./figures/boosted/SigInterpolation/ParamFit_GhhC20_Bukin_Graviton_Para5_rhoR}
\caption{The paramaters of the Bukin functional fit as a function of signal mass points for the graviton c=2.0 signal model.}
\label{fig:boosted_siginter_bukinfitparam_c20}
\end{center}
\end{figure}
\FloatBarrier


\begin{figure}[!h]
\begin{center}
\includegraphics*[width=0.45\textwidth]{./figures/boosted/SigInterpolation/ParamFit_GhhC10RW_Bukin_Graviton_Para0_max}
\includegraphics*[width=0.45\textwidth]{./figures/boosted/SigInterpolation/ParamFit_GhhC10RW_Bukin_Graviton_Para1_x0}\\
\includegraphics*[width=0.45\textwidth]{./figures/boosted/SigInterpolation/ParamFit_GhhC10RW_Bukin_Graviton_Para2_sigma}
\includegraphics*[width=0.45\textwidth]{./figures/boosted/SigInterpolation/ParamFit_GhhC10RW_Bukin_Graviton_Para3_xi}\\
\includegraphics*[width=0.45\textwidth]{./figures/boosted/SigInterpolation/ParamFit_GhhC10RW_Bukin_Graviton_Para4_rhoL}
\includegraphics*[width=0.45\textwidth]{./figures/boosted/SigInterpolation/ParamFit_GhhC10RW_Bukin_Graviton_Para5_rhoR}
\caption{The paramaters of the Bukin functional fit as a function of signal mass points for the graviton c=1.0 signal model.}
\label{fig:boosted_siginter_bukinfitparam_c10}
\end{center}
\end{figure}
\FloatBarrier


\begin{figure}[!h]
\begin{center}
\includegraphics*[width=0.45\textwidth]{./figures/boosted/SigInterpolation/InterpolateShape_GC20_1000}
\includegraphics*[width=0.45\textwidth]{./figures/boosted/SigInterpolation/InterpolateShape_GC20_1500}\\
\includegraphics*[width=0.45\textwidth]{./figures/boosted/SigInterpolation/InterpolateShape_GC20_2000}
\includegraphics*[width=0.45\textwidth]{./figures/boosted/SigInterpolation/InterpolateShape_GC20_3000}
\caption{$m_{hh}$ prediction for several mass points of the graviton c=2.0 signal model. Black histogram
is the prediction from the signal interpolation while the red histogram is the prediction from MC.}
\label{fig:boosted_siginter_templates_c20}
\end{center}
\end{figure}
\FloatBarrier


\begin{figure}[!h]
\begin{center}
\includegraphics*[width=0.45\textwidth]{./figures/boosted/SigInterpolation/InterpolateShape_GC10RW_1000}
\includegraphics*[width=0.45\textwidth]{./figures/boosted/SigInterpolation/InterpolateShape_GC10RW_1500}\\
\includegraphics*[width=0.45\textwidth]{./figures/boosted/SigInterpolation/InterpolateShape_GC10RW_2000}
\includegraphics*[width=0.45\textwidth]{./figures/boosted/SigInterpolation/InterpolateShape_GC10RW_3000}
\caption{$m_{hh}$ prediction for several mass points of the graviton c=1.0 signal model. Black histogram
is the prediction from the signal interpolation while the red histogram is the prediction from MC.}
\label{fig:boosted_siginter_templates_c10}
\end{center}
\end{figure}
\FloatBarrier

\subsection{Expected limits}
\label{app:boosted_siginter_limit}

\begin{figure}[!h]
\begin{center}
\includegraphics*[width=0.49\textwidth]{./figures/boosted/SigInterpolation/Limits_GravitonC20_MCvsInterpolation_StatOnly}
\includegraphics*[width=0.49\textwidth]{./figures/boosted/SigInterpolation/Limits_GravitonC20_MCvsInterpolation_Systs}
\caption{Expected upper limit as a function of the signal Graviton c=2.0 resonance mass. Black curve is the expected
limit with the signal prediction taken directly from MC while the blue curve is the signal prediction obtained from the interpolation.
Plot on the left is the stat-only expected upper limit and on the right is the stat+syst expected upper limit.}
\label{fig:boosted_siginter_templates_limits_c20}
\end{center}
\end{figure}
\FloatBarrier

\begin{figure}[!h]
\begin{center}
\includegraphics*[width=0.49\textwidth]{./figures/boosted/SigInterpolation/Limits_GravitonC10RW_MCvsInterpolation_StatOnly}
\includegraphics*[width=0.49\textwidth]{./figures/boosted/SigInterpolation/Limits_GravitonC10RW_MCvsInterpolation_Systs}
\caption{Expected upper limit as a function of the signal Graviton c=1.0 resonance mass. Black curve is the expected
limit with the signal prediction taken directly from MC while the blue curve is the signal prediction obtained from the interpolation.
Plot on the left is the stat-only expected upper limit and on the right is the stat+syst expected upper limit.}
\label{fig:boosted_siginter_templates_limits_c10}
\end{center}
\end{figure}
\FloatBarrier

% \begin{lstlisting}[language=C++, caption={Bukin fit function C-style function implementation for ROOT TF1}]
% Double_t BukinFunction(Double_t *x, Double_t *par)
% {
%   Double_t xx    = x[0];
%   Double_t max   = par[0]; // max peak
%   Double_t x0    = par[1]; // position of the peak
%   Double_t sigma = par[2]; // width of the core
%   Double_t xi    = par[3]; // asymmetry
%   Double_t rhoL  = par[4]; // size of the lower tail
%   Double_t rhoR  = par[5]; // size of the higher tail

%   Double_t r1 = 0.0;
%   Double_t r2 = 0.0;
%   Double_t r3 = 0.0;
%   Double_t r4 = 0.0;
%   Double_t r5 = 0.0;
%   Double_t hp = 0.0;
  
%   Double_t x1 = 0.0;
%   Double_t x2 = 0.0;
%   Double_t fit_result = 0.0;

%   Double_t consts = 2 * sqrt( 2 * log(2.0) );
%   hp = sigma * consts;
%   r3 = TMath::Log(2.0);
%   r4 = TMath::Sqrt(TMath::Power(xi,2) + 1.0);
%   r1 = xi/r4;

%   if (TMath::Abs(xi) > TMath::Exp(-6.)){
%     r5 = xi / TMath::Log(r4 + xi);
%   }
%   else{
%     r5 = 1.0;
%   }

%   x1 = x0 + (hp / 2) * (r1 - 1);
%   x2 = x0 + (hp / 2) * (r1 + 1);

%   if (xx < x1){
%     //Left Side
%     r2 = rhoL * TMath::Power((xx-x1)/(x0-x1), 2) - r3 + 4 
%          * r3 * (xx-x1) / hp * r5 * r4 / TMath::Power((r4-xi), 2);
%   }
%   else if (xx < x2){
%     //Centre
%     if(TMath::Abs(xi) > TMath::Exp(-6.)){
%       r2 = TMath::Log(1 + 4 * xi * r4 * (xx-x0) / hp ) 
%            / TMath::Log( 1 + 2 * xi * (xi-r4) );
%       r2 = -r3 * (r2*r2);
%     }
%     else{
%       r2 = -4 * r3 * TMath::Power(((xx-x0)/hp), 2);
%     }
%   }
%   else{
%     //Right Side
%     r2 = rhoR * TMath::Power((xx-x2)/(x0-x2), 2) - r3 - 4 
%          * r3 * (xx-x2) / hp * r5 * r4 / TMath::Power((r4+xi), 2);
%   }
    
%   if (TMath::Abs(r2) > 100){
%     fit_result = 0;
%   }
%   else{
%     //Normalize the result
%     fit_result = TMath::Exp(r2);
%   }
%   result = max * fit_result;
%   return result;
% }
% \end{lstlisting}